% $Id$
% Author:	Daniel Bosk <daniel.bosk@miun.se>
\documentclass[addpoints]{miunexam}
\usepackage[utf8]{inputenc}
\usepackage[T1]{fontenc}
\usepackage[swedish,english]{babel}
\usepackage{hyperref,url}
\usepackage{color}
\usepackage{prettyref,varioref}
\usepackage{subfigure}
\usepackage{listings}
\usepackage[binary]{SIunits}
\usepackage[varioref,prettyref,listings]{miunmisc}
\bibliographystyle{alpha}

%\printanswers

\examtype{Final exam}
\courseid{DT011G}
\course{Introduction to Operating Systems}
\date{2013-01-18}

\author{%
	Daniel Bosk\\
  {\small\texttt{\href{mailto:daniel.bosk@miun.se}{daniel.bosk@miun.se}}}\\
  {\small\textit{Phone:} 060\,-\,14\,8709}\\
}

\begin{document}
\maketitle
\thispagestyle{foot}

\section*{Instructions}
\label{sec:Instructions}
\noindent
Carefully read the questions before you start answering them.
Note the time limit of the exam and plan your answers accordingly.
Only answer the question, do not write about subjects remotely related to the
question.

Write your answers on separate sheets, not on the exam paper.
Only write on one side of the sheets.
Start each question on a new sheet.

Make sure you write your answers clearly, if I cannot read an answer the answer
will be awarded no points -- even if the answer is correct.
The questions are \emph{not} sorted by difficulty.

\begin{description}
	\item[Time] 6 hours.
  \item[Aids] Dictionary, course literature \cite{Silberschatz2009osc}, graded 
    assignments and personal notes.
	\item[Maximum points] \numpoints
	\item[Questions] \numquestions
\end{description}

%\subsection*{Bonus points}
%\noindent
%You must get an E or higher, to get the bonus points added to your final grade.
%Bonus points will be added to this exam and the first re-exam.

\subsection*{Preliminary grades}
\noindent
The following grading criteria applies:
E \(\geq 50\%\),
D \(\geq 60\%\),
C \(\geq 70\%\), 
B \(\geq 80\%\),
A \(\geq 90\%\);
with no question awarded zero points.


\section*{Aim}
\label{sec:Aim}
\noindent
The aim of the exam is to examine that you have fulfilled the requirements 
specified in the course syllabus.

\clearpage
\section*{An operating systems oriented computer science fiction,\\
  or: The exam questions}
\noindent

\begin{questions}
  \question[4]\label{q:threads}
  At your work there is this guy Bill, he always comes to you for help.
  He is a nice guy, he always means well.
  But sometimes things just does not turn out as intended, alright, most of the 
  times.
  This time, however, he comes with a minor problem, it is not a problem 
  really, more of a thought.

  ``Sorry to bother you again'', he says, ``I was looking around for ideas on 
  how to improve the built-in web browser in my system'', Bill maintains an 
  operating system in his spare time, ``and I was looking at Firefox and 
  Chromium.
  As I understand it, Firefox runs each tab as a new thread whereas Chromium 
  spawns a new process for each tab.
  Which approach do you think would be the most advantageous?''

  Help Bill by explaining the advantages and disadvantages of each approach.
  \begin{solution}
    % XXX rewrite solution for question on threads
    Threads share memory of a single process \cite[p.  
    153--154]{Silberschatz2009osc}, hence less memory is used to handle many 
    clients.
    This could be the difference between having to swap to disk or not.
    It will definitely be faster to create a thread for each client than to 
    fork an entire process \cite[p. 154]{Silberschatz2009osc} (although the 
    forking in UNIX-like systems is heavily optimized for this).

    The disadvantages include more complex programming (concurrent programming 
    is thus more error prone) which increases the risk of buggy code, which of 
    course can be exploited.
    Another disadvantage is that if a client manages to crash its server, i.e.\ 
    a thread, then the entire web-server goes down -- for all clients!
    E.g.\ generating a memory error in one thread will make the operating 
    system terminate the process, not just a single thread.

    Furthermore, since memory is shared among threads, this means that the 
    thread of one client can access the memory of a thread of another client.
    Thus a client could access the web data of another client, containing e.g.\ 
    credit-card data.
    This is of course provided there exists a vulnerability which can be 
    exploited, but as stated above: multi-threaded programs are more complex 
    and hence more error prone.
  \end{solution}

  \question[5]\label{q:deadlock}
  By the end of the day your telephone rings, it is your not-so-tech-savy 
  friend.
  He is by the way a rather enthusiastic fellow, some would even go for the 
  word excentric.
  His name is Steve, and you are going out with Steve and Bill to catch the 
  latest film in the theatre right after work.

  ``How're you doing'', he says, ``are you and Bill on your way?''

  ``Yes, we'll leave in a few minutes'', you say.

  ``Great!  You know I was thinking about what you said about deadlocks a few 
  months ago.  I'll tell you when you get here.''

  You and Bill leave the office and arrive at the theatre to meet Steve by the 
  registers to get your tickets.

  ``About that deadlock-thingy, I was thinking about getting these tickets.
  You know, the cashier does not trust us to give us the tickets before we pay.
  Hence, we have to trust the cashier by paying and then hoping we get the 
  tickets.
  If we would not trust the cashier, and thus would require the tickets before 
  we pay, \emph{then} we would encounter a deadlock, isn't that right?''

  It would appear Steve actually learned what a deadlock is.
  But just to be sure he is not just lucky, talk him through why this is 
  a deadlock.
  \begin{solution}
    % XXX rewrite solution for deadlock question
    The problem here is not that of a deadlock, it is a common misuse of 
    terminology.
    For a deadlock to occur you need the following four conditions to hold 
    \cite[pp. 285--287]{Silberschatz2009osc}:
    \begin{enumerate}
      \item Mutual exclusion: at least one resource must be held by any process 
        in an unsharable way, such that no other process can access the 
        resource while being held by another process.
      \item Hold and wait: a process must be holding at least one resource and 
        must be waiting for at least one more which is held by another process.
      \item No preemption: there can be no preemption, i.e. there cannot be any 
        process which may interrupt and force another process to release any 
        resource.
        Hence we cannot deallocate a resource from one process to reallocate it 
        to another.
        Thus no kind of priority system may exist.
      \item Circular wait: there must exist a set of processes \(\{P_1, \ldots, 
        P_n\}\) such that \(P_1\) is waiting for \(P_2\), which is waiting for 
        \(P_3\), and so on until \(P_n\) which is waiting for \(P_1\).
    \end{enumerate}
    If these four conditions hold we have a deadlock.

    Then why is our scenario above not a deadlock?
    First, there is no resource for which we have mutual exclusion.

    Second, there is no process holding a resource and waiting for another held 
    by a different process.
    We have one process waiting for another process, the \term{tar(1)} process 
    on the server waiting for data from that on the client, but not in a way 
    required by our condition.

    Third, as there is no resource for which we have mutual exclusion there is 
    no sense in speaking of preemption for any such resource either.

    Fourth, there is no circular wait involved.
    Sure, as stated above the server process waits for data from the client 
    process, but there is no resource which is being held which is waited for.

    It is thus inappropriate use of terminology to call this a deadlock.
    Using an analogy of Sten-Erik Winborg: if you consider the Dining 
    Philosophers Problem, what happened in our scenario is equivalent of 
    removing the bowl in the middle of the table from which the philosophers 
    get their food.
  \end{solution}

  \question\label{q:drivers}
  After the film has ended you pose the question about threads and processes to 
  Steve.

  ``Steve, Bill is thinking about whether to choose threads or processes in his 
  browser, what are your thoughts?''

  ``Certainly he should choose the one easiest for developers.
  Getting developers to your platform is the key to success'', Steve replies.

  ``Sure, but shouldn't I think first about my users'', Bill asks.

  ``Developers, developers, developers'', Steve starts to chant like obsessed, 
  shortly continuing to shouting, ``DEVELOPERS, DEVELOPERS, DEVELOPERS!
  I love that word; some day, if I am to hold a presentation I will start it 
  with those very words.''

  ``We know that for sure, and so does the rest of the theatre.
  I solved the networking stuff by copying the the code from Berkeley, but I do 
  not know how to solve the interface to the hardware.
  I have heard of monolithic kernels, loadable kernel modules and drivers; I do 
  not really know what that means though.''

  You on the other hand, know perfectly well the different approaches to 
  interface the hardware.
  Explain to Bill
  \begin{parts}
    \part[2] the different types of kernels there are, and
    \part[4] what advantages there are of each type.
  \end{parts}
  \begin{solution}
    % XXX write solution to driver question
    \dots
  \end{solution}

  \question\label{q:fs}
  You are at work again.
  Bill is busy working, in the evenings he is implementing the things you have 
  talked about earlier; this is very nice as this allows you to actually get 
  some work done.
  However, you are very good at distracting yourself on your own as well.
  You hear a colleague of yours moaning aloud about how ``it is time to defrag 
  again''.

  Explain to your colleague
  \begin{parts}
    \part[2] what defragmentation is and why it is necessary on some file 
    systems,
    \part[2] tell him about some of the file systems which does not require 
    defragmentation (this includes why they do not need this).
  \end{parts}
  \begin{solution}
    % XXX write solution for fs question
    \dots
  \end{solution}

  \question[6]\label{q:scheduling}
  You are back in your office, it is Friday afternoon.
  Another one of your co-workers comes by.
  He says to you ``I hope you haven't forgotten about our meeting.
  You promised me you'd give me some feedback on my ideas for Freax'', and he 
  smiles.
  He knows just as well as you do that you did forget about it, it was not in 
  your calendar and you have been so distracted this week you remember only 
  fragments.

  ``Well,'' he continues, ``we are supposed to discuss some aspects of 
  scheduling for my free-time project.''
  His name is Linus and he is an intern from the University of Helsinki.
  ``What I'd like to do is to have a nice general scheduling algorithm for 
  processes.
  Interactive processes should have priority over I/O-bound processes''.
  The two of you enter the meeting room.

  Give Linus an overview of how to implement his scheduler.
  (A tip is to look at how the Linux operating system actually does its process 
  scheduling; yes, there is a little ``Back to the future'' about this tip.)
  \begin{solution}
    % XXX write a solution to scheduling question
    The thing which comes to mind is a DFS with location independence.
    This means a system where the physical location of a file can change from 
    one place to another without any visible change in its logical location 
    \cite[p. 707]{Silberschatz2009osc}.
    One system which implements this is AFS.
    To see why this helps our two hippie friends we have to overview the 
    implementation.

    In AFS terminology there are client machines and server machines.
    The servers all present the clients with a homogeneous, identical, and 
    location-transparent file hierarchy -- the shared name space \cite[p.  
    719]{Silberschatz2009osc}.

    More than this AFS features client-side caching with cache consistency 
    \cite[p. 718]{Silberschats2009osc} which means that the client can cache 
    files used often and that these cached versions are guaranteed to be 
    consistent across the system.
    It also includes server-side caching in the form of replicas, with high 
    availability through automatic switchover to a replica if the source server 
    is temporaily unavailable \cite[p. 718]{Silberschatz2009osc}.
    (It also provides authentication using Kerberos.)
  \end{solution}
\end{questions}

Finally Friday evening, now it is time for a treat after all this hard work.
\begin{center}
  \textbf{The end.}
\end{center}

\bibliography{literature}
\end{document}
