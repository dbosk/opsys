% $Id$
% Author:	Daniel Bosk <daniel.bosk@miun.se>
\documentclass[addpoints]{miunexam}
\usepackage[utf8]{inputenc}
\usepackage[T1]{fontenc}
\usepackage[swedish,english]{babel}
\usepackage{hyperref,url}
\usepackage{color}
\usepackage{prettyref,varioref}
\usepackage{subfigure}
\usepackage{listings}
\usepackage[binary]{SIunits}
\usepackage[varioref,prettyref,listings]{miunmisc}
\bibliographystyle{alpha}

\printanswers

\examtype{Final exam}
\courseid{DT011G}
\course{Introduction to Operating Systems}
\date{2013-01-18}

\author{%
	Daniel Bosk\\
  {\small\texttt{\href{mailto:daniel.bosk@miun.se}{daniel.bosk@miun.se}}}\\
  {\small\textit{Phone:} 060\,-\,14\,8709}\\
}

\begin{document}
\maketitle
\thispagestyle{foot}
\tableofcontents

\section*{Instructions}
\label{sec:Instructions}
\noindent
Carefully read the questions before you start answering them.
Note the time limit of the exam and plan your answers accordingly.
Only answer the question, do not write about subjects remotely related to the
question.

Write your answers on separate sheets, not on the exam paper.
Only write on one side of the sheets.
Start each question on a new sheet.

Make sure you write your answers clearly, if I cannot read an answer the answer
will be awarded no points -- even if the answer is correct.
The questions are \emph{not} sorted by difficulty.

\begin{description}
	\item[Time] 6 hours.
  \item[Aids] Dictionary, course literature, graded assignments and personal 
    notes.
	\item[Maximum points] \numpoints
	\item[Questions] \numquestions
\end{description}

%\subsection*{Bonus points}
%\noindent
%You must get an E or higher, to get the bonus points added to your final grade.
%Bonus points will be added to this exam and the first re-exam.

\subsection*{Preliminary grades}
\noindent
E \(\geq 50\%\) with no question having zero points,
D \(\geq 60\%\),
C \(\geq 70\%\), 
B \(\geq 80\%\),
A \(\geq 90\%\).


\section*{Aim}
\label{sec:Aim}
\noindent
The aim of the exam is to examine that you have fulfilled the requirements 
specified in the course syllabus.

\clearpage
\section*{An excerpt from the life of a systems administrator,\\
  or: the exam questions}
\noindent

\begin{questions}
  \question[4]\label{q:apache}
  A colleague at your work-place is setting up a web-server.
  He is working with the Apache web-server.
  The Apache web-server is modular by design, meaning it has loadable modules 
  to get different behaviour on different systems.
  For instance, it has one module which spawns a child process for each client 
  to handle its requests.
  Another of Apaches modules creates a thread for each client to handle its 
  requests.

  Your colleague asks you: ``What are the advantages and disadvantages of each 
  approach?  Which one should I pick?''

  \begin{solution}
    Threads share memory of a single process \cite[p.  
    153--154]{Silberschatz2009osc}, hence less memory is used to handle many 
    clients.
    This could be the difference between having to swap to disk or not.
    It will definitely be faster to create a thread for each client than to 
    fork an entire process \cite[p. 154]{Silberschatz2009osc} (although the 
    forking in UNIX-like systems is heavily optimized for this).

    The disadvantages include more complex programming (concurrent programming 
    is thus more error prone) which increases the risk of buggy code, which of 
    course can be exploited.
    Another disadvantage is that if a client manages to crash its server, i.e.\ 
    a thread, then the entire web-server goes down -- for all clients!
    E.g.\ generating a memory error in one thread will make the operating 
    system terminate the process, not just a single thread.

    Furthermore, since memory is shared among threads, this means that the 
    thread of one client can access the memory of a thread of another client.
    Thus a client could access the web data of another client, containing e.g.\ 
    credit-card data.
    This is of course provided there exists a vulnerability which can be 
    exploited, but as stated above: multi-threaded programs are more complex 
    and hence more error prone.
  \end{solution}

  \question[6]\label{q:deadlock}
  After work you are visiting a friend, this friend of yours is not very 
  tech-savy (although he himself thinks that he is).
  When you arrive he says to you: ``You know what, I managed to deadlock a ssh 
  and tar pipe the other day.''
  You look strangely at him as he continues, ``well, I ran \term{tar -zcf 
  - <files> | ssh user@server tar -zxf -},'' a trick you taught him, ``and then 
  I accidentaly disconnected my access point.  The program was deadlocked for 
  several minutes before it finally quit.''

  Explain to your friend what a deadlock really is and why this is not 
  a deadlock.

  \begin{solution}
    The problem here is not that of a deadlock, it is a common misuse of 
    terminology.
    For a deadlock to occur you need the following four conditions to hold 
    \cite[pp. 285--287]{Silberschatz2009osc}:
    \begin{enumerate}
      \item Mutual exclusion: at least one resource must be held by any process 
        in an unsharable way, such that no other process can access the 
        resource while being held by another process.
      \item Hold and wait: a process must be holding at least one resource and 
        must be waiting for at least one more which is held by another process.
      \item No preemption: there can be no preemption, i.e. there cannot be any 
        process which may interrupt and force another process to release any 
        resource.
        Hence we cannot deallocate a resource from one process to reallocate it 
        to another.
        Thus no kind of priority system may exist.
      \item Circular wait: there must exist a set of processes \(\{P_1, \ldots, 
        P_n\}\) such that \(P_1\) is waiting for \(P_2\), which is waiting for 
        \(P_3\), and so on until \(P_n\) which is waiting for \(P_1\).
    \end{enumerate}
    If these four conditions hold we have a deadlock.

    Then why is our scenario above not a deadlock?
    First, there is no resource for which we have mutual exclusion.

    Second, there is no process holding a resource and waiting for another held 
    by a different process.
    We have one process waiting for another process, the \term{tar(1)} process 
    on the server waiting for data from that on the client, but not in a way 
    required by our condition.

    Third, as there is no resource for which we have mutual exclusion there is 
    no sense in speaking of preemption for any such resource either.

    Fourth, there is no circular wait involved.
    Sure, as stated above the server process waits for data from the client 
    process, but there is no resource which is being held which is waited for.

    It is thus inappropriate use of terminology to call this a deadlock.
    Using an analogy of Sten-Erik Winborg: if you consider the Dining 
    Philosophers Problem, what happened in our scenario is equivalent of 
    removing the bowl in the middle of the table from which the philosophers 
    get their food.
  \end{solution}

  \question[6]\label{q:addrspace}
  A little embaressed for his misunderstanding of what a deadlock is (remember, 
  he considers himself as tech-savy) he quickly changes the conversation.
  ``Oh, I have to tell you, I just upgraded my legacy 32-bit hardware for my 
  in-closet server to a bit more modern 64-bit hardware.''
  Whereby you reply: ``Nice, did you know something cool, the 64-bit address 
  space is so large that hierarchical paging systems are in general too 
  ineffective to use, one has to resolve to hashed or inverted page tables 
  instead.''
  Your friend looks at you in disbelief and says: ``And why is that?
  How can it be effective for 32-bit systems but not for 64-bit systems?  That 
  makes no sense.''

  How do you answer him?
  (A tip is to give a convincing example and calculating the effective access 
  time for that example, there is no arguing against that as long as it is 
  a valid example and your calculations are correct.)

  \begin{solution}
    First, let us assume that each memory access takes 
    \unit{100}{\nano\second}.
    Let us further assume that we use 32-bit physical addresses in a 32-bit 
    system and 64-bit physical addresses in a 64-bit system.
    Also assume a page size of \unit{4}{\kibi\byte}, i.e.\ \(2^{12}\) bytes.

    For clarity we make no use of TLBs.

    In a 32-bit system, we use hierarchical paging because otherwise the page 
    table of each process would be too large:
    we would need contigous memory to hold \(2^{32} / 2^{12} = 2^{20}\) 
    entries, each of which being 32 bits, or 4 bytes.
    That is, we would need \unit{4}{\mebi\byte} of contigous memory.
    This would counter the benefits of using paged memory and is thus 
    unacceptable.
    When we introduce hierachical paging we want to fit page tables into single 
    pages (and corresponding frames) \cite[p. 337]{Silberschatz2009osc}.
    By dividing the 20 bits of page number into two 10-bit parts we can use the 
    first part (most significant bits) as the outer page.
    There are \(2^{10}\) entries which will fit nicely into a single page.
    Each of these entries contain a page number, we use the inner page number 
    as an offset into this page.
    The address at this offset is another page, and this is the actual page 
    which we are looking for.
    Notice that we need to perform three memory references.
    One for the outer page, one for the inner page, and then the final one to 
    access the actual page.
    With our assumptions this would mean that the effective access time will be 
    \(\unit{100}{\nano\second} + \unit{100}{\nano\second} 
    + \unit{100}{\nano\second} = \unit{300}{\nano\second}.\)
    In the single-level paging scheme we first access the page table and then 
    the page, yielding an effective access time of only 
    \unit{200}{\nano\second}.

    Looking at the 64-bit system we realise that we would need a six-level page 
    table: \(2^{64} / 2^{12} = 2^{52}\) entries to store, i.e.\ 
    \unit{4}{\tebi\byte} of memory is required to store the page table.
    For this we need five 10-bit outer pages and one 2-bit outer page.
    This would require a total of seven memory accesses and hence would yield 
    an effective access time of \unit{700}{\nano\second} -- not as slow as 
    a hard-disk drive, but we are closing in.
    And alas, we must find another way to do this, we need a lower effective 
    access time.
  \end{solution}

  \question[3]\label{q:mountpoint}
  Obviously your friend has not taken a course on operating systems concepts.
  And quite frankly you begin to see his jealousy, especially as he continues:  
  ``Well, since you seem to know it all, can you also explain why my files in 
  \term{/pub} disappears when I try to mount my external harddrive in that 
  directory as well?  I would like to see both the files in the directory and 
  those on my external harddrive in there simultaneously.''

  Explain to your friend why this is not possible.

  \begin{solution}
    As this is commonly implemented a directory contains directory entries, 
    each corresponding to a file or subdirectory and has a flag indicating 
    whether it is a mount point or not \cite[pp.  
    444--445]{Silberschatz2009osc}.
    We could allow the mount-point flag to be set and not ignore the directory 
    entries, but here follows why we do not do this.
    
    The problem becomes obvious when you ask yourself the question \emph{In 
    which file system would we create a new file?}.
		Changing existing files is not a problem, as the operating system can open 
		them for reading and writing.
    Although, both directories, one in each file system, would have to be 
    searched for the file.
		The problem appears when the user wants to create a non-existent file, the 
		operating system cannot possibly know in which directory to create an entry 
		for the file.
		As the mounting of a file system is transparent to the user the user does 
		not even know that there are two possible file systems and thus cannot 
		notify the operating system in any way which file system to use.

		A possible solution would be to always create new files in the directory of 
		the file system using the other directory as a mount point.
		Since we did mount another file system here it is reasonable to assume we 
		are interested in using it; but this is, of course, not always true.

    A further problem is: what happens if the same filename occurs in both 
    directories?
    Because of the transparency for the user the operating system cannot make 
    a difference between these two files.
  \end{solution}

  \question[6]\label{q:dfs}
  The next day, back in your work-place, you see two strangers, a man and 
  a woman seeming a little lost.
  To you they look like two stiff hippies walking about the IT-department, 
  feeling very out-of-place.
  You see them speaking to one of your colleagues and he points towards you.
  They come over to you and the woman starts talking:  ``We heard that you are 
  the one to talk to about distributed file system infrastructure.''
  You look back at them with a smile as this is one of your favourite topics.
  The man continues, ``well, you see, we want to connect our Sundsvall, 
  Stockholm, London and Sydney offices with a shared file system.  But I say 
  this is not a good idea as the file server would probably end up in London or 
  Sydney, and it is slow as it is already when it is in the same building.''
  He adds ``I do editing of raw film material, which is quite large files, and 
  me and my colleagues share this material in our network-mapped home 
  directories right now.  I fear that with this system each read and write 
  would require data to travel to Sydney and back again.''
  The woman interrupts him by saying: ``I have tried to convince him that there 
  are solutions to this, that there are more advanced file systems than this.  
  You know the details better than me, please tell him.''

  Explain to them how you would solve the problem to ensure performance to ease 
  his worries, and also how your solution could provide some sort of redundancy 
  in case of a short temporary network failure between any two offices.

  \begin{solution}
    The thing which comes to mind is a DFS with location independence.
    This means a system where the physical location of a file can change from 
    one place to another without any visible change in its logical location 
    \cite[p. 707]{Silberschatz2009osc}.
    One system which implements this is AFS.
    To see why this helps our two hippie friends we have to overview the 
    implementation.

    In AFS terminology there are client machines and server machines.
    The servers all present the clients with a homogeneous, identical, and 
    location-transparent file hierarchy -- the shared name space \cite[p.  
    719]{Silberschatz2009osc}.

    More than this AFS features client-side caching with cache consistency 
    \cite[p. 718]{Silberschats2009osc} which means that the client can cache 
    files used often and that these cached versions are guaranteed to be 
    consistent across the system.
    It also includes server-side caching in the form of replicas, with high 
    availability through automatic switchover to a replica if the source server 
    is temporaily unavailable \cite[p. 718]{Silberschatz2009osc}.
    (It also provides authentication using Kerberos.)
  \end{solution}
\end{questions}

At last, it is finally Friday evening.
\begin{center}
  \textbf{The end.}
\end{center}

\bibliography{literature}
\end{document}
