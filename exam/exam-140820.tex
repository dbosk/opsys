% $Id$
% Author:	Daniel Bosk <daniel.bosk@miun.se>
\documentclass[addpoints,svv]{miunexam}
\usepackage[utf8]{inputenc}
\usepackage[T1]{fontenc}
\usepackage[swedish,english]{babel}
\usepackage{hyperref,url}
\usepackage{color}
\usepackage{prettyref,varioref}
\usepackage{subfigure}
\usepackage{listings}
\usepackage[binary]{SIunits}
\usepackage[varioref,prettyref,listings]{miunmisc}
\bibliographystyle{alpha}

%\printanswers

\examtype{Final exam}
\courseid{DT011G}
\course{Introduction to Operating Systems}
\date{2014-08-20}

\author{%
	Daniel Bosk\\
  {\small\texttt{\href{mailto:daniel.bosk@miun.se}{daniel.bosk@miun.se}}}\\
  {\small\textit{Phone:} 060\,-\,14\,8709}\\
}

\begin{document}
\maketitle
\thispagestyle{foot}

\section*{Instructions}
\label{sec:Instructions}
Carefully read the questions before you start answering them.
Note the time limit of the exam and plan your answers accordingly.
Only answer the question, do not write about subjects remotely related to the
question.

Write your answers on separate sheets, not on the exam paper.
Only write on one side of the sheets.
Start each question on a new sheet.

Make sure you write your answers clearly, if I cannot read an answer the answer
will be awarded no points -- even if the answer is correct.
The questions are \emph{not} sorted by difficulty.

Note that your answers for this exam does \emph{not} have to include references 
to relevant literature.

\begin{description}
	\item[Time] 5 hours.
  \item[Aids] Dictionary,
    course literature 
    \cite{Silberschatz2009osc,Silberschatz2013osc,Silberschatz2013intl},
    graded assignments,
    personal notes,
    and a calculator.
    The student may use the following e-book reader:
    TrekStor eBook-reader 3.0.
	\item[Maximum points] \numpoints
	\item[Questions] \numquestions
\end{description}

%\subsection*{Bonus points}
%\noindent
%You must get an E or higher, to get the bonus points added to your final grade.
%Bonus points will be added to this exam and the first re-exam.

\subsection*{Preliminary grades}

To pass this exam you need to have at least an average of one (1) point per 
question, and no question may be awarded zero points.
The averages for the grades can be summarised as follows:
E \(\geq 1\),
D \(\geq 1.5\),
C \(\geq 2\),
B \(\geq 2.5\),
A \(\geq 3\),
with no question awarded zero points.


\section*{Aim}
\label{sec:Aim}
The aim of the exam is to examine that you have fulfilled the requirements 
specified in the course syllabus.

\clearpage
\section*{An operating systems oriented computer science fiction,\\
  or: The exam questions}

\begin{questions}
  \question[3]\label{q:apache}
  A colleague at your work-place is setting up a web-server.
  He is working with the Apache web-server.
  The Apache web-server is modular by design, meaning it has loadable modules 
  to get different behaviour on different systems.
  For instance, it has one module which spawns a child process for each client 
  to handle its requests.
  Another of Apache's modules creates a thread for each client to handle its 
  requests.

  Your colleague asks you: ``What are the advantages and disadvantages of each 
  approach?  Which one should I pick?''

  \begin{solution}
    Threads share memory of a single process \cite[p.  
    153--154]{Silberschatz2009osc}, hence less memory is used to handle many 
    clients.
    This could be the difference between having to swap to disk or not.
    It will definitely be faster to create a thread for each client than to 
    fork an entire process \cite[p. 154]{Silberschatz2009osc} (although the 
    forking in UNIX-like systems is heavily optimized for this).

    The disadvantages include more complex programming (concurrent programming 
    is thus more error prone) which increases the risk of buggy code, which of 
    course can be exploited.
    Another disadvantage is that if a client manages to crash its server, i.e.\ 
    a thread, then the entire web-server goes down -- for all clients!
    E.g.\ generating a memory error in one thread will make the operating 
    system terminate the process, not just a single thread.

    Furthermore, since memory is shared among threads, this means that the 
    thread of one client can access the memory of a thread of another client.
    Thus a client could access the web data of another client, containing e.g.\ 
    credit-card data.
    This is of course provided there exists a vulnerability which can be 
    exploited, but as stated above: multi-threaded programs are more complex 
    and hence more error prone.
  \end{solution}


  \question[3]\label{q:deadlock}
  After work you are visiting a friend, this friend of yours is not very 
  tech-savy (although he himself thinks that he is).
  When you arrive he says to you: ``You know what, I managed to deadlock a ssh 
  and tar pipe the other day.''
  You look strangely at him as he continues, ``well, I ran \term{tar -zcf 
  - <files> | ssh user@server tar -zxf -},'' a trick you taught him, ``and then 
  I accidentaly disconnected my access point.  The program was deadlocked for 
  several minutes before it finally quit.''

  Explain to your friend what a deadlock really is and why this is not 
  a deadlock.

  \begin{solution}
    The problem here is not that of a deadlock, it is a common misuse of 
    terminology.
    For a deadlock to occur you need the following four conditions to hold 
    \cite[pp. 285--287]{Silberschatz2009osc}:
    \begin{enumerate}
      \item Mutual exclusion: at least one resource must be held by any process 
        in an unsharable way, such that no other process can access the 
        resource while being held by another process.
      \item Hold and wait: a process must be holding at least one resource and 
        must be waiting for at least one more which is held by another process.
      \item No preemption: there can be no preemption, i.e. there cannot be any 
        process which may interrupt and force another process to release any 
        resource.
        Hence we cannot deallocate a resource from one process to reallocate it 
        to another.
        Thus no kind of priority system may exist.
      \item Circular wait: there must exist a set of processes \(\{P_1, \ldots, 
        P_n\}\) such that \(P_1\) is waiting for \(P_2\), which is waiting for 
        \(P_3\), and so on until \(P_n\) which is waiting for \(P_1\).
    \end{enumerate}
    If these four conditions hold we have a deadlock.

    Then why is our scenario above not a deadlock?
    First, there is no resource for which we have mutual exclusion.

    Second, there is no process holding a resource and waiting for another held 
    by a different process.
    We have one process waiting for another process, the \term{tar(1)} process 
    on the server waiting for data from that on the client, but not in a way 
    required by our condition.

    Third, as there is no resource for which we have mutual exclusion there is 
    no sense in speaking of preemption for any such resource either.

    Fourth, there is no circular wait involved.
    Sure, as stated above the server process waits for data from the client 
    process, but there is no resource which is being held which is waited for.

    It is thus inappropriate use of terminology to call this a deadlock.
    Using an analogy of Sten-Erik Winborg: if you consider the Dining 
    Philosophers Problem, what happened in our scenario is equivalent of 
    removing the bowl in the middle of the table from which the philosophers 
    get their food.
  \end{solution}


  \question[3]\label{q:addrspace}
  A little embaressed for his misunderstanding of what a deadlock is (remember, 
  he considers himself as tech-savy) he quickly changes the conversation.
  ``Oh, I have to tell you, I just upgraded my legacy 32-bit hardware for my 
  in-closet server to a bit more modern 64-bit hardware.''
  Whereby you reply: ``Nice, did you know something cool, the 64-bit address 
  space is so large that hierarchical paging systems are in general too 
  ineffective to use, one has to resolve to hashed or inverted page tables 
  instead.''
  Your friend looks at you in disbelief and says: ``And why is that?
  How can it be effective for 32-bit systems but not for 64-bit systems?  That 
  makes no sense.''

  How do you answer him?
  (A tip is to give a convincing example and calculating the effective access 
  time for that particular example, there is no arguing against that as long as 
  it is a valid example and your calculations are correct.)

  \begin{solution}
    First, let us assume that each memory access takes 
    \unit{100}{\nano\second}.
    Let us further assume that we use 32-bit physical addresses in a 32-bit 
    system and 64-bit physical addresses in a 64-bit system.
    Also assume a page size of \unit{4}{\kibi\byte}, i.e.\ \(2^{12}\) bytes.

    For clarity we make no use of TLBs.

    In a 32-bit system, we use hierarchical paging because otherwise the page 
    table of each process would be too large:
    we would need contigous memory to hold \(2^{32} / 2^{12} = 2^{20}\) 
    entries, each of which being 32 bits, or 4 bytes.
    That is, we would need \unit{4}{\mebi\byte} of contigous memory.
    This would counter the benefits of using paged memory and is thus 
    unacceptable.
    When we introduce hierachical paging we want to fit page tables into single 
    pages (and corresponding frames) \cite[p. 337]{Silberschatz2009osc}.
    By dividing the 20 bits of page number into two 10-bit parts we can use the 
    first part (most significant bits) as the outer page.
    There are \(2^{10}\) entries which will fit nicely into a single page.
    Each of these entries contain a page number, we use the inner page number 
    as an offset into this page.
    The address at this offset is another page, and this is the actual page 
    which we are looking for.
    Notice that we need to perform three memory references.
    One for the outer page, one for the inner page, and then the final one to 
    access the actual page.
    With our assumptions this would mean that the effective access time will be 
    \(\unit{100}{\nano\second} + \unit{100}{\nano\second} 
    + \unit{100}{\nano\second} = \unit{300}{\nano\second}.\)
    In the single-level paging scheme we first access the page table and then 
    the page, yielding an effective access time of only 
    \unit{200}{\nano\second}.

    Looking at the 64-bit system we realise that we would need a six-level page 
    table: \(2^{64} / 2^{12} = 2^{52}\) entries to store, i.e.\ 
    \unit{4}{\tebi\byte} of memory is required to store the page table.
    For this we need five 10-bit outer pages and one 2-bit outer page.
    This would require a total of seven memory accesses and hence would yield 
    an effective access time of \unit{700}{\nano\second} -- not as slow as 
    a hard-disk drive, but we are closing in.
    And alas, we must find another way to do this, we need a lower effective 
    access time.
  \end{solution}


  \question
  Back at work in the morning, you are trying to focus on your work.
  However, one of your colleagues comes to you with Silberschatz in hand.

  ``We add paging to primary memory to remove problems of (external) 
  fragmentation,'' she says.
  ``In the same manner we divide the secondary memory into blocks (and 
  clusters).
  Then why do we talk about fragmentation in file systems when it's not an 
  issue in RAM?''

  You're thinking to yourself, ``well, that's not true of all file systems, but 
  fine, we're not on that level yet.''

  \begin{parts}
    \part[3] Explain what is meant by ``fragmentation'' when we talk about 
    files in secondary memory.
    \begin{solution}
      What we mean by fragmentation is that data is fragmented.
      We have two types of fragmentation, external and internal fragmentation.
      External being an issue due to contiguous allocation, because then free 
      space might be fragmented so that there is enough free space -- but it's 
      not contiguous and thus cannot be used.

      Internal fragmentation arises when we try to counter external 
      fragmentation.
      We counter external fragmentation by dividing the space into blocks of 
      a fixed size.
      This way, whenever we need some space we allocate a sufficient amount of 
      blocks.
      In some cases we need less space than a full block, hence some space (the 
      leftovers inside the block) is wasted.
      But this way we can use all unallocated blocks as long as there are 
      enough blocks free to cover our need.

      This solution is applied both to primary and secondary storage.
      In primary memory we have paging to map a contiguous logical memory to 
      a non-contiguous physical memory.
      For secondary storage, this means a contiguous file (logical space) is 
      not stored contiguously in the storage medium (physical space).
      Hence, what we mean by ``fragmentation'' in this case is that the blocks 
      of the file are not stored contiguously but spread across the disk.
    \end{solution}

    \part[3] Explain why this can be an issue in secondary memory but not in 
    primary despite us using the same methods in both places.
    \begin{solution}
      The reason this ``fragmentation'' is an issue in secondary storage and 
      not in primary storage is due to the type of storage medium.
      In primary storage, physical memory, we can access all parts of the 
      physical memory module equally fast.
      In secondary storage, a magnetic disk, we cannot do this.
      This is due to the fact that a magnetic disk rotates and has a reading 
      arm which has to be moved into position and then wait for the disk to 
      rotate to the correct position.
      Thus, the time it takes to read from this disk depends on how we read the 
      data.
      If the blocks of a file are stored in consecutive order on the disk we 
      can read them all together, if not, then we have to move the read head 
      and wait for rotation.

      As such, this is no longer an issue in secondary storage with solid state 
      disks, this is because these work more like the physical memory modules 
      of primary storage.
    \end{solution}
  \end{parts}


  \question[3]
  You are finally approaching the end of your day.
  Thinking back on a day full of interruptions, you start thinking ``oh, how 
  I wish I could just turn off preemption sometimes.''
  That reminds you that OSes sometimes turns off interrupts in kernel mode to 
  avoid race conditions, ``why couldn't Mother Nature make the world more like 
  an x86-architecture?''

  That the kernel must protect its data structures by locks in a multiprocessor 
  environment is obvious, but why must the kernel do that in a single processor 
  environment too?
  \begin{solution}
    The problem we address in this question is about locks in the kernel -- not 
    processes.
    The kernel is one running program, in this case running only on a single 
    processor (single core).
    The reason we need to use locking mechanisms to protect sensitive kernel 
    data structures is due to interrupts.
    Although the kernel executes only in one thread it can be interrupted at 
    any time, and the interrupt handler also executes as kernel code.
    This way the kernel might execute in two critical sections simultaneously.

    One solution is to introduce locks, another is to turn off interrupts in 
    the entry section of a critical section and then turn it on again in the 
    exit section.
  \end{solution}

\end{questions}

Finally you are done, go through your answers thoroughly again(!) and then go 
for a treat after all this hard work.
\begin{center}
  \textbf{The end.}
\end{center}

\bibliography{literature}
\end{document}
