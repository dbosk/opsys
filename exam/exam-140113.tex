% $Id$
% Author:	Daniel Bosk <daniel.bosk@miun.se>
\documentclass[addpoints]{miunexam}
\usepackage[utf8]{inputenc}
\usepackage[T1]{fontenc}
\usepackage[swedish,english]{babel}
\usepackage{hyperref,url}
\usepackage{color}
\usepackage{prettyref,varioref}
\usepackage{subfigure}
\usepackage{listings}
\usepackage[binary]{SIunits}
\usepackage[varioref,prettyref,listings]{miunmisc}
\bibliographystyle{alpha}

%\printanswers

\examtype{Final exam}
\courseid{DT011G}
\course{Introduction to Operating Systems}
\date{2014-01-13}

\author{%
	Daniel Bosk\\
  {\small\texttt{\href{mailto:daniel.bosk@miun.se}{daniel.bosk@miun.se}}}\\
  {\small\textit{Phone:} 060\,-\,14\,8709}\\
}

\begin{document}
\maketitle
\thispagestyle{foot}

\section*{Instructions}
\label{sec:Instructions}
\noindent
Carefully read the questions before you start answering them.
Note the time limit of the exam and plan your answers accordingly.
Only answer the question, do not write about subjects remotely related to the
question.

Write your answers on separate sheets, not on the exam paper.
Only write on one side of the sheets.
Start each question on a new sheet.

Make sure you write your answers clearly, if I cannot read an answer the answer
will be awarded no points -- even if the answer is correct.
The questions are \emph{not} sorted by difficulty.

Note that your answers for this exam does \emph{not} have to include references 
to relevant literature.

\begin{description}
	\item[Time] 5 hours.
  \item[Aids] Dictionary,
    course literature 
    \cite{Silberschatz2009osc,Silberschatz2013osc,Silberschatz2013intl},
    graded assignments,
    personal notes,
    and a calculator.
    The student may use the following e-book reader:
    TrekStor eBook-reader 3.0.
	\item[Maximum points] \numpoints
	\item[Questions] \numquestions
\end{description}

%\subsection*{Bonus points}
%\noindent
%You must get an E or higher, to get the bonus points added to your final grade.
%Bonus points will be added to this exam and the first re-exam.

\subsection*{Preliminary grades}
\noindent
The following grading criteria applies:
E \(\geq 50\%\),
D \(\geq 60\%\),
C \(\geq 70\%\), 
B \(\geq 80\%\),
A \(\geq 90\%\);
with no question awarded zero points.


\section*{Aim}
\label{sec:Aim}
\noindent
The aim of the exam is to examine that you have fulfilled the requirements 
specified in the course syllabus.

\clearpage
\section*{An operating system saga,\\
  or: The exam questions}
\noindent

\begin{questions}
  \question
  It's an early Monday morning, you've just arrived to your office.
  You're thinking to yourself ``Oh, this is too early for me, I shouldn't get 
  out of bed before noon.''
  Considering this you decide to go to the fika room to see if any of your 
  colleagues feel the same way, and in which case you can feel sorry for each 
  other.
  One of your colleagues is there, sitting in the sofa with a copy of 
  Silberschatz's \emph{Operating System Concepts} in his lap.
  He looks up at you and says ``You're knowledgable in this, I don't understand 
  this kernel and user mode.
  What's the point of those anyway?''

  Give a short description of
  \begin{parts}
    \part[3] why we need to make the distinction of kernel and user mode, and
    \begin{solution}
      We need to make this distinction because otherwise a simple application 
      program could take over the roll of the operating system (OS), e.g.\ it 
      could simply skip giving control back to the OS.
      So the OS needs kernel and user mode to be able to enforce its rules on 
      the application programs.
      (See \cite[pp.\ 21-22]{Silberschatz2009osc} for details.)
    \end{solution}

    \part[3] the requirements to implement the two modes.
    \begin{solution}
      The OS runs in kernel mode, it sets a timer (hardware) to generate an 
      interrupt in 50 milliseconds, then it changes to user mode and starts the 
      execution of application code.
      After 50 milliseconds the timer generates an interrupt, this sets the 
      mode back to kernel mode and executes the OS's interrupt handler.
      The control is now back with the OS.
      For this to work it must be a privileged instruction to set the timer, we 
      need a hardware timer to generate an interrupt.

      We also need support from hardware for the two modes, a mode bit must 
      exist in hardware so that the processor knows what kind of code it 
      executes, and in the case of executing a privileged instruction in user 
      mode it will generate a trap.
    \end{solution}
  \end{parts}


  \question
  Feeling much better after a soft start of the day you return to your desk.
  You start thinking about the CPU scheduler in the OS of your laptop computer.
  The scheduler is very good most of the times, but sometimes things feel like 
  their very slow.

  Your train of thought goes down a tricky path.
  \begin{parts}
    \part[3] Explain why not all OSes use the optimal scheduling algorithm 
    shortest-job-first (SJF)?
    \begin{solution}
      SJF is optimal in that it gives the lowest possible average waiting time.
      However, this requires it to know all future jobs and their CPU-burst 
      lengths, something which is impossible.
      We can at best estimate this.

      Due to its design, SJF also gives rise to the possibility of starvation.
      Some processes might never execute because processes with shorter 
      CPU-bursts are run before.

      Also, some OSes optimize the scheduler for other things than minimizing 
      the average waiting time.
    \end{solution}

    \part[3] Also explain what is a good scheduling strategy for an ordinary 
    laptop computer system.
    \begin{solution}
      An ordinary laptop will run mostly interactive programs, hence we want it 
      to be responsive.
      To achive this we can use many ready-queues instead of only one, and 
      processes can move between queues depending on their execution behaviour.
      This method is called multi-level feedback queue.

      This way, we can favour I/O-intensive processes, since these will use the 
      CPU a shorter time before relinquishing the CPU due to I/O.
      Then CPU-intensive processes can use the CPU when all I/O-bound processes 
      are waiting for I/O.
      This way we will get a responsive system since it's the I/O-bound 
      processes which will be interactive ones.
    \end{solution}
  \end{parts}


  \question
  Interrupted from your thoughts on scheduling, your colleague from the morning 
  is back again -- still with Silberschatz in hand.

  ``We add paging to primary memory to remove problems of (external) 
  fragmentation,'' she says.
  ``In the same manner we divide the secondary memory into blocks (and 
  clusters).
  Then why do we talk about fragmentation in file systems when it's not an 
  issue in RAM?''

  You're thinking to yourself, ``well, that's not true of all file systems, but 
  fine, we're not on that level yet.''

  \begin{parts}
    \part[3] Explain what is meant by ``fragmentation'' when we talk about 
    files in secondary memory.
    \begin{solution}
      What we mean by fragmentation is that data is fragmented.
      We have two types of fragmentation, external and internal fragmentation.
      External being an issue due to contiguous allocation, because then free 
      space might be fragmented so that there is enough free space -- but it's 
      not contiguous and thus cannot be used.

      Internal fragmentation arises when we try to counter external 
      fragmentation.
      We counter external fragmentation by dividing the space into blocks of 
      a fixed size.
      This way, whenever we need some space we allocate a sufficient amount of 
      blocks.
      In some cases we need less space than a full block, hence some space (the 
      leftovers inside the block) is wasted.
      But this way we can use all unallocated blocks as long as there are 
      enough blocks free to cover our need.

      This solution is applied both to primary and secondary storage.
      In primary memory we have paging to map a contiguous logical memory to 
      a non-contiguous physical memory.
      For secondary storage, this means a contiguous file (logical space) is 
      not stored contiguously in the storage medium (physical space).
      Hence, what we mean by ``fragmentation'' in this case is that the blocks 
      of the file are not stored contiguously but spread across the disk.
    \end{solution}

    \part[3] Explain why this can be an issue in secondary memory but not in 
    primary despite us using the same methods in both places.
    \begin{solution}
      The reason this ``fragmentation'' is an issue in secondary storage and 
      not in primary storage is due to the type of storage medium.
      In primary storage, physical memory, we can access all parts of the 
      physical memory module equally fast.
      In secondary storage, a magnetic disk, we cannot do this.
      This is due to the fact that a magnetic disk rotates and has a reading 
      arm which has to be moved into position and then wait for the disk to 
      rotate to the correct position.
      Thus, the time it takes to read from this disk depends on how we read the 
      data.
      If the blocks of a file are stored in consecutive order on the disk we 
      can read them all together, if not, then we have to move the read head 
      and wait for rotation.

      As such, this is no longer an issue in secondary storage with solid state 
      disks, this is because these work more like the physical memory modules 
      of primary storage.
    \end{solution}
  \end{parts}


  \question[3]
  You are finally approaching the end of your day.
  Thinking back on a day full of interruptions, you start thinking ``oh, how 
  I wish I could just turn off preemption sometimes.''
  That reminds you that OSes sometimes turns off interrupts in kernel mode to 
  avoid race conditions, ``why couldn't Mother Nature make the world more like 
  an x86-architecture?''

  That the kernel must protect its data structures by locks in a multiprocessor 
  environment is obvious, but why must the kernel do that in a single processor 
  environment too?
  \begin{solution}
    The problem we address in this question is about locks in the kernel -- not 
    processes.
    The kernel is one running program, in this case running only on a single 
    processor (single core).
    The reason we need to use locking mechanisms to protect sensitive kernel 
    data structures is due to interrupts.
    Although the kernel executes only in one thread it can be interrupted at 
    any time, and the interrupt handler also executes as kernel code.
    This way the kernel might execute in two critical sections simultaneously.

    One solution is to introduce locks, another is to turn off interrupts in 
    the entry section of a critical section and then turn it on again in the 
    exit section.
  \end{solution}


\end{questions}

Finally, now it is time for a treat after all this hard work.
\begin{center}
  \textbf{The end.}
\end{center}

\bibliography{literature}
\end{document}
