% $Id$
% Author:	Daniel Bosk <daniel.bosk@miun.se>
\documentclass[addpoints,svv]{miunexam}
\usepackage[utf8]{inputenc}
\usepackage[T1]{fontenc}
\usepackage[swedish,english]{babel}
\usepackage{hyperref,url}
\usepackage{color}
\usepackage{prettyref,varioref}
\usepackage{subfigure}
\usepackage{listings}
\usepackage[binary]{SIunits}
\usepackage[varioref,prettyref,listings]{miunmisc}
\bibliographystyle{alpha}

%\printanswers

\examtype{Final exam}
\courseid{DT011G}
\course{Introduction to Operating Systems}
\date{2014-04-24}

\author{%
	Daniel Bosk\\
  {\small\texttt{\href{mailto:daniel.bosk@miun.se}{daniel.bosk@miun.se}}}\\
  {\small\textit{Phone:} 060\,-\,14\,8709}\\
}

\begin{document}
\maketitle
\thispagestyle{foot}

\section*{Instructions}
\label{sec:Instructions}
Carefully read the questions before you start answering them.
Note the time limit of the exam and plan your answers accordingly.
Only answer the question, do not write about subjects remotely related to the
question.

Write your answers on separate sheets, not on the exam paper.
Only write on one side of the sheets.
Start each question on a new sheet.

Make sure you write your answers clearly, if I cannot read an answer the answer
will be awarded no points -- even if the answer is correct.
The questions are \emph{not} sorted by difficulty.

Note that your answers for this exam does \emph{not} have to include references 
to relevant literature.

\begin{description}
	\item[Time] 5 hours.
  \item[Aids] Dictionary,
    course literature 
    \cite{Silberschatz2009osc,Silberschatz2013osc,Silberschatz2013intl},
    graded assignments,
    personal notes,
    and a calculator.
    The student may use the following e-book reader:
    TrekStor eBook-reader 3.0.
	\item[Maximum points] \numpoints
	\item[Questions] \numquestions
\end{description}

%\subsection*{Bonus points}
%\noindent
%You must get an E or higher, to get the bonus points added to your final grade.
%Bonus points will be added to this exam and the first re-exam.

\subsection*{Preliminary grades}

To pass this exam you need to have at least an average of one (1) point per 
question, and no question may be awarded zero points.
The averages for the grades can be summarised as follows:
E \(\geq 1\),
D \(\geq 1.5\),
C \(\geq 2\),
B \(\geq 2.5\),
A \(\geq 3\),
with no question awarded zero points.


\section*{Aim}
\label{sec:Aim}
The aim of the exam is to examine that you have fulfilled the requirements 
specified in the course syllabus.

\clearpage
\section*{An operating systems oriented computer science fiction,\\
  or: The exam questions}

\begin{questions}
  \question[3]\label{q:threads}
  Your colleague at work, Bill, who always comes to you for help, comes for 
  your aid today again.
  He is a nice guy, he always means well.
  But sometimes things just does not turn out as intended, alright, most of the 
  times.
  This time, however, he with an actual problem.

  ``Sorry to bother you again'', he says, ``I was thinking about threads and 
  processes.''
  As we know from earlier, Bill maintains an operating system in his spare 
  time.
  ``I know I have talked to you about this previously too, but I've forgotten 
  some of what you said, and I have some new ideas'', he says.

  ``OK, shoot'', you say.

  ``Well, I'm working on handling processes and threads.
  I've looked at how UNIX-like systems does this, but I'm not quite sure 
  I understand it completely.  Can you explain it to me?''

  Help Bill by explaining how to approach process and thread handling.
  More specifically, tell Bill about:
  \begin{parts}
    \part[3] The advantages of the Linux implementation of the fork(2) system 
    call.
    \part[3] How the copy-on-write technique works, and why it is good.
    \part[3] The different ways to handle threads and their advantages.
  \end{parts}
  \begin{solution}
    % XXX rewrite solution for question on threads
    \dots
  \end{solution}

  \question[3]\label{q:deadlock}
  By the end of the day your telephone rings, it is your not-so-tech-savy 
  friend.
  He is by the way a rather enthusiastic fellow, some would even go for the 
  word excentric.
  It is your friend Steve, and you are going out with Steve and Bill to catch 
  the latest film in the cinema right after work, you do this every now and 
  then.

  ``How're you doing'', he says, ``are you and Bill on your way?''

  ``Yes, we'll leave in a few minutes, Linus might also join us'', you say.

  ``Great!  You know I was thinking about what you said about deadlocks a few 
  months ago.  I'll tell you when you get here.''

  You and Bill leave the office and arrive at the cinema to meet Steve by the 
  registers to get your tickets.

  ``Linus couldn't join us, he was busy hacking on some features in Linux,'' 
  you say.

  ``I've never understood why he publishes his code for free on the net'', 
  Steve says.
  ``He should do like Bill and start to charge people money for it.''

  ``Well, I think it's quite good'', Bill says, ``this way I can copy some of 
  his features and incorporate it into my Wall-Hole operating system.  I've 
  done that from the Berkely Software Distribution.''

  ``About that deadlock-thingy, I was thinking about that on my way from work 
  the other day.  I was in my car, heading for a round-about.
  I was thinking, this could turn into a deadlock situation.  Don't you 
  think?''

  Discuss with Steve whether this is a deadlock or not.
  \begin{solution}
    % XXX rewrite solution for deadlock question
    \dots
  \end{solution}

  \question\label{q:drivers}
  After the film has ended you pose the question about threads and processes to 
  Steve.

  ``Steve, Bill is thinking about whether to choose threads or processes in his 
  browser, what are your thoughts?''

  ``Certainly he should choose the one easiest for developers.
  Getting developers to your platform is the key to success'', Steve replies.

  ``Sure, but shouldn't I think first about my users'', Bill asks.

  ``Developers, developers, developers'', Steve starts to chant like obsessed, 
  shortly continuing to shouting, ``DEVELOPERS, DEVELOPERS, DEVELOPERS!
  I love that word; some day, if I am to hold a presentation I will start it 
  with those very words.''

  ``We know that for sure, and so does the rest of the cinema.
  I solved the networking stuff by copying the the code from BSD, but I do not 
  know how to solve the interface to the hardware.
  I have heard of monolithic kernels, loadable kernel modules and drivers; I do 
  not really know what that means though.''

  You on the other hand, know perfectly well the different approaches to 
  interface the hardware.
  Explain to Bill
  \begin{parts}
    \part[3] the different types of kernels there are (how they work), and
    \part[3] what advantages there are of each type.
  \end{parts}
  \begin{solution}
    % XXX write solution to driver question
    \dots
  \end{solution}

  \question[3]\label{q:fs}
  Back in your office trying to work, your boss comes by your desk.
  ``I've got a wee bit o'problem, laddie,'' he says, ``it concerns the 
  efficiency of page replacement algorithms.''

  ``Mhm, go on,'' you say.

  ``Well, it's like this, I've got this page-reference string'', he shows you 
  a paper containing a string of numbers:
  \begin{quote}
    1, 2, 3, 4, 1, 1, 2, 4, 5, 5, 6, 8, 7, 8, 8, 9, 9, 9, 9, 1
  \end{quote}
  He continues, ``a memory access takes \unit{100}{\nano\second} if the page is 
  in memory, with a page-fault it takes \unit{10}{\milli\second} to load the 
  page into memory and another \unit{100}{\nano\second} to complete the 
  access.''

  You nod and say, ``and you want me to tell you the efficiency of some 
  algorithms compared to the optimal page-replacement?''

  His eyes are shining, ``exactly, laddie!'' he says enthusiastically.
  ``It's the second-chance algorithm, there is only a concern of reading, no 
  writing done to memory.  How does this algorithm perform compared to the 
  optimal page-replacement algorithm?  We've got only five wee frames to work 
  with, lad.''

  Do the following calculations and explain them thoroghly:
  \begin{parts}
    \part[3] What is the average access time for the optimal page-replacement 
    algorithm for this reference string?

    \begin{solution}
      \dots
    \end{solution}

    \part[3] What is the average access time for the second-chance 
    page-replacement algorithm for this reference string?

    \begin{solution}
      \dots
    \end{solution}

  \end{parts}

  \question[3]\label{q:scheduling}
  You are back in your office, it is Friday afternoon.
  Another one of your co-workers comes by.
  He says to you ``I hope you haven't forgotten about our meeting.
  You promised me you'd give me some feedback on my ideas for Freax'', and he 
  smiles.
  He knows just as well as you do that you did forget about it, it was not in 
  your calendar and you have been so distracted this week you remember only 
  fragments.

  ``Well,'' he continues, ``we are supposed to discuss some aspects of 
  scheduling for my free-time project.''
  It is your friend Linus, who didn't join you for cinema, and he is actually 
  an intern from the University of Helsinki.
  ``What I'd like to do is to have a nice general scheduling algorithm for 
  processes.
  Interactive processes should have priority over I/O-bound processes''.
  The two of you enter the meeting room.

  Give Linus an overview of how to implement his scheduler.
  \begin{solution}
    % XXX write a solution to scheduling question
    \dots
  \end{solution}
\end{questions}

Finally Thursday evening, now it is time for a treat after all this hard work.
\begin{center}
  \textbf{The end.}
\end{center}

\bibliography{literature}
\end{document}
