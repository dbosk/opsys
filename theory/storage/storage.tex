% $Id$
% Author: Daniel Bosk <daniel.bosk@miun.se>
\documentclass[a4paper]{miunasgn}
\usepackage[utf8]{inputenc}
\usepackage[swedish,english]{babel}
\usepackage{url,hyperref}
\usepackage{prettyref,varioref}
\usepackage[longnamesfirst]{natbib}
\usepackage[binary,amssymb,noams]{SIunits}
\usepackage[nofancy,today]{svninfo}
\usepackage[natbib,varioref,prettyref]{miunmisc}

\svnInfo $Id$

%\printanswers

\courseid{DT141G}
\course{Operating Systems}
\assignmenttype{Theory assignment}
\title{Storage}
\author{Daniel Bosk\footnote{%
	This work is licensed under the Creative Commons Attribution-ShareAlike 3.0 
	Unported license.
	To view a copy of this license, visit 
	\url{http://creativecommons.org/licenses/by-sa/3.0/}.
}}
\date{\svnId}

\begin{document}
\maketitle
\thispagestyle{foot}
\tableofcontents


\section{Aim}
\label{sec:Aim}
The aim of the assignment is, first, to aid your understanding of the treated 
content by providing questions and problems which inspire reflection.
Second, it is to examinate the following:
\begin{itemize}
	% $Id$
% Author:	Daniel Bosk <daniel.bosk@miun.se>
\item That you can analyze the problems faced by a scheduling algorithm and 
other problems in a time-sharing environment.
\item That you know how to solve these problems, i.e.\ know some common 
scheduling algorithms.
\item That you can analyze a situation and choose among scheduling algorithms 
and what advantages or disadvantages certain adaptions to them might have.
\item That you can analyze a situation with regard to synchronization and apply 
a correct solution to avoid race conditions and deadlocks.

\end{itemize}


\section{Prerequisites}
\label{sec:Prerequisites}
% $Id$
% Author:	Daniel Bosk <daniel.bosk@miun.se>
This assignment covers part three of the book 
\cite{Silberschatz2013intl,Silberschatz2013osc}, i.e.\ memory management.
Hence, before attempting this assignment you should have read chapters 8 and 9, 
``Memory-Management Strategies'' and ``Virtual-Memory Management'', 
respectively.



\section{Tasks}
\label{sec:Work}
\begin{questions}
	\question\label{q:file}
	In the context of computer file-systems, define the term \emph{file}.
	\begin{solution}
		A \emph{file} is a named logical storage-unit, which the operating system 
		uses to abstract the usage of its storage system \cite[p.  
		421]{Silberschatz2009osc}.
		The files are then mapped onto a physical storage-device.

		\citeauthor{Silberschatz2009osc} continue by saying that ``a file is the 
		smallest allotment of logical secondary storage'' \cite[p.  
		422]{Silberschatz2009osc}.
		The file thus contains information, the organization of which is up to the 
		creator of the file.

		The above is sufficient for answering the question.

		Apart from contents, a file has both attributes and operations \citep[pp.  
		422--423]{Silberschatz2009osc}.
		Some of the attributes a file can have are
		\begin{itemize}
			\item a human-readable name,
			\item an identifier in the file system,
			\item a location of the file on a device,
			\item a size,
			\item some access-control information, and
			\item time-stamps when the file was created, last accessed, etc.
		\end{itemize}

		To complete the definition of file, we must state the operations which can 
		be performed on a file.
		According to \citet[p. 423]{Silberschatz2009osc} these operations are (at 
		least)
		\begin{itemize}
			\item creating a new file,
			\item writing to a file,
			\item reading from a file,
			\item seeking to a given position within a file,
			\item deleting an existing file, and finally
			\item truncating an existing file discarding all contents but keeping the 
				attributes.
		\end{itemize}
	\end{solution}

	\question\label{q:directories}
	Define the terms
	\begin{parts}
		\part \emph{directory},
		\part \emph{path}, and
		\part \emph{search path}.
	\end{parts}
	\begin{solution}
		A \emph{directory} is a logical part of the file system.
		It is used to ease the organization of files.
		Each directory contains a list of all files it contains \cite[p.  
		435]{Silberschatz2009osc}.
		This list also includes subdirectories of that directory \cite[p.  
		439]{Silberschatz2009osc}.

		A \emph{path}, or \emph{path name}, is a directory traversal from the root, 
		in the case of an \emph{absolute} path, or from the current working 
		directory, in the case of a \emph{relative} path, to a given file or 
		directory in the file system \cite[p.  439]{Silberschatz2009osc}.
		For each directory on this path, starting from the beginning, the operating 
		system opens the directory entry and finds the next directory or file in 
		the path and then proceeds.

		The \emph{search path} is a list of path names to directories which are 
		searched by the shell when a command is given \cite[p.  
		438]{Silberschatz2009osc}.
	\end{solution}

	\question\label{q:mounting}
	When a file system is mounted onto a mount point which is a directory already 
	containing files, those files will no longer be visible until the file system 
	is unmounted again \cite[pp. 444--445]{Silberschatz2009osc}.
	Why is that?
	\begin{solution}
		The problem becomes obvious when you ask yourself the question \emph{In 
		which file system would we create a new file?}.
		Changing existing files is not a problem, as the operating system can open 
		them for reading and writing.
		Although, both directories, one in each file system, have to be searched 
		for the file.
		The problem appears when the user wants to create a non-existent file, the 
		operating system cannot possibly know in which directory to create an entry 
		for the file.
		As the mounting of a file system is transparent to the user the user does 
		not even know that there are two possible file systems and thus cannot 
		notify the operating system in any way which file system to use.

		A possible solution would be to always create new files in the directory of 
		the file system using the other directory as a mount point.
		Since we did mount another file system here it is reasonable to assume we 
		are interested in using it, but this is, of course, not always true.
	\end{solution}

	\question\label{q:filesystems}
	You have a system which groups the disk blocks into \unit{4}{\kibi\byte} 
	sized clusters.
	You are given the following two file systems.

	First, a FAT-based file system.
	The FAT maps clusters and not physical disk blocks.
	It uses 32-bit numbers for the clusters.

	Second, an inode-based file system.
	An inode contains 16 pointers, each of which is 32 bits in size.
	The first 13 point directly to clusters.
	The next points to a single indirect block, i.e.\ a cluster of direct 
	pointers.
	The next-to-last points to a double indirect block and the last points to 
	a triple indirect block.
	\begin{parts}
		\part Find the maximum size of a single file in each file system.
		\part How much space is wasted on meta data in each file system?
	\end{parts}
	Remember to get the units right.
	\begin{solution}
		In the FAT-based file system the maximum size of a file is \[
			2^{32}\cdot 4096 = 2^{32}\cdot 2^{12} = 2^{44},
		\] that is \unit{16}{\tebi\byte}.
		This means that one file has allocated all clusters.

		As the FAT-scheme requires one pointer per cluster \cite[cf. p.  
		475]{Silberschatz2009osc} the space required for meta data in the 
		FAT-scheme is \(2^{32}\cdot 4 = \unit{16}{\gibi\byte}.\)

		In the inode-based file system the maximum size of a file is given by the 
		sum of the following three equations.
		For the first 14 direct pointers we have
		\begin{equation}
			\label{eq:direct}
			14\cdot 4096 = 7\cdot 2^3\cdot 2^{10} = \unit{56}{\kibi\byte}.
		\end{equation}
		For the single indirect pointers we have an entire cluster of direct 
		pointers, thus
		\begin{equation}
			\label{eq:singleindirect}
			\frac{4096}{4}\cdot 4096 = 2^{10}\cdot 2^{12} = 2^{22} 
			= \unit{4}{\mebi\byte}
		\end{equation}
		of data.
		For the double indirect pointers we get as much space for direct pointers 
		as data \prettyref{eq:singleindirect}, hence we get
		\begin{equation}
			\label{eq:doubleindirect}
			\frac{2^{22}}{4}\cdot 4096 = 2^{20}\cdot 2^{12} = 2^{32} 
			= \unit{4}{\gibi\byte}
		\end{equation}
		of data here.
		Similarily, using the triple indirect pointers we get
		\begin{equation}
			\label{eq:tripleindirect}
			\frac{2^{32}}{4}\cdot 4096 = 2^{30}\cdot 2^{12} = 2^{42} 
			= \unit{4}{\tebi\byte}.
		\end{equation}
		Adding these equations together yields a maximum file size of just above 
		\unit{4}{\tebi\byte}.

		The space required for meta data in the inode-scheme is \[
			16\cdot 4 + 2^{12} + ( 2^{12} + 2^{22} ) + ( 2^{12} + 2^{22} + 2^{32} 
			) < 2^{33} = \unit{8}{\gibi\byte},
		\] which is way less than \(4\cdot 2^{32}\) as in the FAT-scheme.
	\end{solution}

	\question\label{q:ssd}
	Given a solid-state disk (SSD), which disk-scheduling algorithm would you 
	like your operating system to use for I/O-scheduling?
	The available algorithms in your operating system is FCFS, SSTF, SCAN, 
	C-SCAN, LOOK, and C-LOOK.
	Motivate your choice well.
	\begin{solution}
		As all blocks on an SSD disk are accessed equally fast, independent of 
		previously accessed blocks, the FCFS is the best choice.
		The FCFS is the most fair among the algorithms.

		All other algorithms are adapted to disks with non-constant access time 
		\cite[cf. sec. 12.4]{Silberschatz2009osc}, which includes seek time and 
		rotational latency, this is not the case for SSD disks.
	\end{solution}

	\question\label{q:raid}
	You are tasked to set up the server storing the home folders for all users in 
	the company.
	Each user's home folder will then be mounted using a DFS, perhaps NFS.
	The characteristics for the users vary from editing film material (huge 
	files!) to simple text-file editing and web-browser caching.
	There are at the moment approximately 200 users, but the company is 
	continously expanding.
	At your disposal you have 20 disks of \unit{2}{\tera\byte} each and 
	accompanying host-bus adapters.
	A backup is performed each night, but if a user has to wait for a restore 
	from backup you will be hung by your feet in the foyer of the company 
	building -- these backups are only for saving the users from their own 
	mistakes.
	Furthermore, if the system is too slow, the users will be frustrated and also 
	hang you by your feet.
	How do you make use of the disks?
	\begin{solution}
		Obviously we need some kind of RAID-based solution.
		We need to account for both performance and security.
		We will need both high write-performance and read-performance.

		Judging from section 12.7.3 in \cite{Silberschatz2009osc}, RAID level 1+0 
		is a good choice.
		However, RAID levels 5 and 6 are also accepted.
	\end{solution}
\end{questions}


%%% EXAMINATION %%%
% $Id$
% Author:	Daniel Bosk <daniel.bosk@miun.se>
\section{Examination}
\label{sec:Examination}
\noindent
Your answers should be handed in using the course platform.
They should be well-written using correct references including page or section 
numbers\footnote{%
	This is also to better prepare you for the exam, then you can use your 
	assignment to quickly find information in the book.
}.


\bibliography{literature}
\end{document}
