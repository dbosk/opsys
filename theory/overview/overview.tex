% $Id$
% Author: Daniel Bosk <daniel.bosk@miun.se>
\documentclass[a4paper,nocourse]{miunasgn}
\usepackage[utf8]{inputenc}
\usepackage[swedish,english]{babel}
\usepackage{url,hyperref}
\usepackage{prettyref,varioref}
\usepackage{natbib}
\usepackage[nofancy,today]{svninfo}
\usepackage[natbib,varioref,prettyref]{miunmisc}

\svnInfo $Id$

%\printanswers

\courseid{DT141G}
\course{Operating Systems}
\assignmenttype{Theory assignment}
\title{Overview}
\author{Daniel Bosk\footnote{%
	This work is licensed under the Creative Commons Attribution-ShareAlike 3.0 
	Unported license.
	To view a copy of this license, visit 
	\url{http://creativecommons.org/licenses/by-sa/3.0/}.
	Some of the questions are derived from the work of 
	\citeauthor*{Silberschatz2009osc}.
}}
\date{\svnId}

\begin{document}
\maketitle
\thispagestyle{foot}
\tableofcontents


\section{Aim}
\label{sec:Aim}
The aim of the assignment is, first, to aid your understanding of the treated 
content by providing questions and problems which inspire reflection.
Second, it is to examinate the following:
\begin{itemize}
	% $Id$
% Author:	Daniel Bosk <daniel.bosk@miun.se>
\item That you can analyze the problems faced by a scheduling algorithm and 
other problems in a time-sharing environment.
\item That you know how to solve these problems, i.e.\ know some common 
scheduling algorithms.
\item That you can analyze a situation and choose among scheduling algorithms 
and what advantages or disadvantages certain adaptions to them might have.
\item That you can analyze a situation with regard to synchronization and apply 
a correct solution to avoid race conditions and deadlocks.

\end{itemize}


\section{Prerequisites}
\label{sec:Prerequisites}
% $Id$
% Author:	Daniel Bosk <daniel.bosk@miun.se>
This assignment covers part three of the book 
\cite{Silberschatz2013intl,Silberschatz2013osc}, i.e.\ memory management.
Hence, before attempting this assignment you should have read chapters 8 and 9, 
``Memory-Management Strategies'' and ``Virtual-Memory Management'', 
respectively.



\section{Tasks}
\label{sec:Work}
\begin{questions}
	\question\label{q:ostypes}
	Define the following operating system types:
	\begin{parts}
		\part batch,
		\part interactive,
		\part time-sharing,
		\part realtime, and
		\part distributed.
	\end{parts}
	\begin{solution}
		\begin{description}
			\item[Batch] Jobs are executed in bulk, e.g. batch(1) jobs on modern 
				UNIX-like systems.
				As the jobs are executed in bulk and then the results are returned to 
				the user these systems do not wait for user input, all input to the job 
				must be supplied from the beginning.

				Performance can be increased by grouping together jobs which will keep 
				both CPU and I/O devices busy.
				This can be accomplished by e.g. buffering and multiprogramming.

			\item[Interactive] This type of system executes in short bursts, commonly 
				in the magnitude of seconds, and then the user provides input for the 
				next burst of execution.
				The results of these bursts are usually hard to predict as they depend 
				on user input.
				
			\item[Time-sharing] This system executes a job for a very short period of 
				time and then switches to another.
				It iterates through all jobs, and does this so quickly the user 
				conceives it as all processes running simultaenously.

			\item[Realtime] A realtime operating system operates under timely 
				restrictions.
				All jobs must be executed within a given timeframe, if not they fail 
				and other routines must be run to make up for the error -- if possible.
				Examples include systems handling traffic lights or some kind of 
				machinery.

			% XXX look over answer for distributed systems
			\item[Distributed] A distributed operating system has its functionality 
				and parts distributed over several systems.
				
		\end{description}
	\end{solution}
	
	\question\label{q:multiprogramming}
	Describe the advantages of multiprogramming.
	\begin{solution}
		The advantage of multiprogramming is that when one process is idly waiting 
		for some operation to complete, e.g. I/O, then the operating system can 
		switch to another process for use of the CPU.
		When the operation the first process was waiting for has completed the 
		operating system can switch back and continue its execution.
		This way the use of the resources of the system is maximized.

		Some disadvantages (which are not asked for in this question), however, 
		include that the system has to handle more than one process simultanously.  
		This implies more memory is needed to have all processes in memory, if not, 
		it requires swapping to swap them to a swapping device.
	\end{solution}

	\question\label{q:privinstr}
	Describe why or why not the following instructions should be either 
	privileged or not privileged:
	\begin{parts}
		\part setting a timer,
		\part reading the time from the system clock,
		\part freeing (or erasing) memory,
		\part turning off interrupt handling,
		\part switching between user and kernel mode.
	\end{parts}
	\begin{solution}
		Setting a timer should be a privileged instruction as the system uses 
		timers together with interrupts to handle preemptive scheduling.
		Hence turning interrupt handling off should also be a privileged 
		instruction.
		
		Reading the system clock does not have to be privileged as it is 
		a read-only instruction and contains no secret information.

		Freeing or erasing memory should be a privileged instruction as not to 
		allow a process to free the memory of another process.
		On the other hand, a process should be allowed to free its own memory by 
		a call to the operating system, and this call should of course not be 
		privileged.

		Switching between user and kernel mode must be a privileged instruction.
		Otherwise, to execute a privileged instruction a process only needs to 
		switch to kernel mode and then execute it (and then switch back to user 
		mode).
	\end{solution}

	\question\label{q:interrupts}
	Define the terms
	\begin{parts}
		\part trap, and
		\part interrupt.
	\end{parts}
	(What is the difference between them?)
	\begin{solution}
		An interrupt is a hardware signal to the processor which makes it interrupt 
		what it is currently doing and execute one of its interrupt handling 
		routines, depending on which interrupt occured.

		Similarly a trap is also an interrupt, but it is generated by a software 
		error rather than hardware.

		An interrupt can be generated by a harddrive signalling it is done with 
		I/O, whereas the trap can be generated by a division by zero.
	\end{solution}

	\question\label{q:kernelshell}
	Define the terms
	\begin{parts}
		\part kernel, and
		\part shell.
	\end{parts}
	\begin{solution}
		The kernel is the main part of an operating system.
		This is usually where e.g. functionality for handling memory and file 
		systems reside.
		It is the kernel which manages processes and their memory.

		The shell is what separates the user from the kernel, the shell is 
		a program which the user can interact with to make the operating system 
		(i.e.\ kernel) do what is wishes, e.g. starting an editor process and then 
		the editor process uses system calls to the kernel to open files.
	\end{solution}

	\question\label{q:kernelshellcntd}
	Describe the relationship of the kernel and shell in
	\begin{parts}
		\part a UNIX-like operating system (not including MacOS), and
		\part the Microsoft Windows operating system.
	\end{parts}
	\begin{solution}
		The kernel and shell in a traditional UNIX-like operating system are 
		visibly separated and a specific shell can be used for each user.
		This means that the shell can exchanged for another.
		Examples of traditional UNIX shells are the Bourne shell and Korn shell.
		The shell in a UNIX-like operating system can also be graphical, i.e.\ using 
		the X Window System.

		In the Microsoft Windows operating system there is one shell which is 
		tightly integrated with the operating system and thus cannot easily be 
		exchanged for another.
	\end{solution}

	\question\label{q:os}
	Describe the role of an operating system in a computer system, including its 
	relationship to the other parts of the system.
	\begin{solution}
		Something like it being in between the user and the hardware \dots
	\end{solution}

\end{questions}


%%% EXAMINATION %%%
% $Id$
% Author:	Daniel Bosk <daniel.bosk@miun.se>
\section{Examination}
\label{sec:Examination}
\noindent
Your answers should be handed in using the course platform.
They should be well-written using correct references including page or section 
numbers\footnote{%
	This is also to better prepare you for the exam, then you can use your 
	assignment to quickly find information in the book.
}.



\bibliography{literature}
\end{document}
