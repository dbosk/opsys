% $Id$
% Author: Daniel Bosk <daniel.bosk@miun.se>
\documentclass[a4paper,logo]{miunart}
\usepackage[utf8]{inputenc}
\usepackage{prettyref,varioref}
\usepackage[swedish,english]{babel}
\usepackage[hyphens]{url}
\usepackage{hyperref}
\usepackage[today,nofancy]{svninfo}
\usepackage{listings}
\usepackage{booktabs}
\usepackage[natbib,style=numeric-comp,maxbibnames=99,sorting=none,backend=bibtex8]{biblatex}
\addbibresource{literature.bib}
\usepackage[prettyref,varioref,listings]{miunmisc}

\svnInfo $Id$

\title{The Complete Studyguide for the Course\\
  DT141G Operating Systems}
\author{Daniel Bosk\footnote{%
  This work is licensed under the Creative Commons Attribution-ShareAlike 3.0 
  Unported license.
  To view a copy of this license, visit 
  \url{http://creativecommons.org/licenses/by-sa/3.0/}.
}}
\date{\svnId}

\begin{document}
\maketitle


\section{Aims}

The overviewing aim of the course is to give the student an understanding of 
how computer systems work, this includes all systems from a mobile phone or 
router to a high-performance server.
Focus is on the operating system which connects hardware with software and 
ultimately with its human users.

More specifically, after successfully completing this course you should have 
fulfilled the following aims:
\begin{itemize}
%  \input{overview-aim.tex}
%  \input{process-aim.tex}
%  \input{memory-aim.tex}
%  \input{paging-aim.tex}
%  \input{storage-aim.tex}
%  \input{distributed-aim.tex}
  \input{exam-aim.tex}
\end{itemize}


\section{Course overview}

The book used for this course is \citetitle{Silberschatz2013intl} by 
\citet{Silberschatz2013intl}.
This book provides a sufficient foundation for students in Computer Science.
This course covers most parts of the book.
Note that this book differs a little from another edition of the same book.
The one above is the ``International Student Version'', and is a cheaper 
version of the book.
However, in this course we will also talk about two chapters from the standard 
edition \cite{Silberschatz2013osc}.
These two chapters are Chapter 16 ``Virtual Machines'' and Chapter 17 
``Distributed Systems''.
These areas are interesting for you, but we will not include them in the 
examination.

The aims of the course are examined using a written exam.
There are assignments of both theoretical and laboratory nature, although most 
of them are theoretical, which are for your learning of the material.
They do help you to prepare for the exam.
There is a strong correlation between students who pass the exam and students 
who have done the assignments before taking the exam.

%However, there are some parts of chapter 14 which can be skipped.
%In chapter 14 you need only read sections 14.1--14.8.
%Chapter 15 \emph{System Security} can be read lightly, the content of this 
%chapter will be the focus in other courses.
%Chapters 19--22 covering special purpose systems and case studies may be read 
%lightly as well.

The assignments are numbered starting from zero, this number is prefixed with 
`T' for theoretical and `L' for laboratory assignments.
Each assignment instruction, for both laboratory and theoretical ones, contains 
reading instructions.
These reading instructions are also given here, in the following sections, 
along with the reading instructions for the lectures.

The only assignment which is mandatory is T0, this assignment must be handed-in 
within the first three weeks of the course.
See \prettyref{sec:exam} for details.
The other assignments are awarded bonus points if handed in and passed before 
the deadline.
After the deadline the assignment will be covered during a tutoring session, 
and hence no solutions will be accepted for submission after that.

The lectures aim to give an overview of the topic, to ease your reading of the 
material, and to focus on the parts usually found most difficult by students.
This means you will still \emph{have to read the book} even if you attend the 
lectures.
The study stategy recommended for this course is to read the material during 
the corresponding week, attend the corresponding lecture, then do the 
assignment.
Should any problems arise, then you bring those up during one of the tutoring 
sessions.

A timetable for the course is given in \prettyref{tbl:Timetable}.
The table is adapted for a study rate at 50 percent and should be used as an 
indicator of when to have read what.

\begin{table}
  \centering
  \begin{tabular}{cp{10cm}}
    \toprule
    \textbf{Course Week} & \textbf{Teaching and Course Work}\\
    \toprule
    1
        & Introductory lecture.\\
        & Lecture on system overview.\\
        & T0 Overview.\\
        & Tutoring session.\\
    \midrule
    2
        & Lecture on system structure.\\
        & Tutoring session.\\
    \midrule
    3
        & Lectures on processes, part I.\\
        & Lectures on processes, part II.\\
        & Lectures on processes, part III.\\
        & Lectures on processes, part IV.\\
        & T1 Processes.\\
        & Tutoring session.\\
    \midrule
    4
        & Lectures on memory, part I.\\
        & T2 Memory.\\
        & Lectures on memory, part II.\\
%        & Lectures on memory, part III.\\
        & L3 Paging Algorithms.\\
        & Tutoring session.\\
    \midrule
    5
        & Lectures on storage, part I.\\
        & T4 Storage.\\
        & Lectures on storage, part II.\\
%        & Lectures on storage, part III.\\
%        & Lectures on storage, part IV.\\
%        & L5 Disk Algorithms.\\
        & Tutoring session.\\
    \midrule
    6
        & Lecture on distributed systems.\\
%        & Lectures on distributed systems, part I.\\
%        & T5 Distributed Systems.\\
%        & Lectures on distributed systems, part II.\\
%        & Lectures on distributed systems, part III.\\
        & Lecture on various operating systems.\\
        & Tutoring session.\\
    \midrule
    7
        & Lecture on virtual machines.\\
%        & T6 Virtual Machines.\\
%        & Lecture on protection and security.\\
        & Tutoring session.\\
    \midrule
    8
        & Independent study for exam.\\
    \midrule
    9
        & Independent study for exam.\\
        & Final tutoring session.\\
    \midrule
    10
        & Final exam.\\
    \bottomrule
  \end{tabular}
  \caption{A timetable for the course given at a study rate of 50 percent.}
  \label{tbl:Timetable}
\end{table}

\subsection{Introductory lecture}
\input{intro-lit.tex}

\subsection{Lectures on system overview and system structure}
\input{struct-lit.tex}

\subsection{T0 Overview}
\input{overview-lit.tex}

\subsection{Lectures on process management}
\begin{enumerate}
  \item \input{proc-lit.tex}

  \item \input{thread-lit.tex}

  \item \input{sched-lit.tex}

  \item \input{sync-lit.tex}
\end{enumerate}

\subsection{T1 Processes}
\input{process-lit.tex}

\subsection{Lectures on memory management}
\begin{enumerate}
  \item \input{physmem-lit.tex}

  \item \input{vmem-lit.tex}
\end{enumerate}

\subsection{T2 Memory}
\input{memory-lit.tex}

\subsection{L3 Paging Algorithms}
\input{paging-lit.tex}

\subsection{Lectures on storage management}
\begin{enumerate}
  \item \input{introfs-lit.tex}

  \item \input{lectfs-lit.tex}

%  \item \input{stor-lit.tex}
%
%  \item \input{io-lit.tex}
\end{enumerate}

\subsection{T4 Storage}
\input{storage-lit.tex}

%\subsection{L5 Disk Algorithms}
%\input{fs-lit.tex}

\subsection{Lecture on distributed systems}
\input{distsys-lit.tex}

%\subsection{T6 Distributed Systems}
%\input{distributed-lit.tex}

\subsection{Lecture on various operating systems}
\input{systems-lit.tex}

\subsection{Lecture on virtual machines}
\input{vm-lit.tex}

%\subsection{T7 Virtual Machines}
%\input{vmtheory-lit.tex}


\section{Examination}
\label{sec:exam}
As noted above the course is examined using an introductory theory assignment, 
``T0 Overview'', and a written exam.
The theory assignment is graded pass (P) or fail (F) and is reported as I101 in 
Ladok.
The final exam is graded A, B, C, D, E for pass or F, Fx for fail.
The exam is reported as T101 in Ladok, and it corresponds to 7.5 credit points.

To finish the course you need to pass the above and the grade of the exam will 
be the final grade on the course total.

% XXX rethink the bonus system
Each completed assignment, which is correct and handed in before deadline, will 
be awarded one bonus point for the exam.


\section{What if I'm not done in time?}
\label{sec:late}
The deadlines on this course are of great importance.
You must have completed the first assignment, \emph{T0 Overview}, within its 
deadline.
If you do not do this you will be deregistered from the course and your place 
will be open to other applicants.

The other theory and laboratory assignments must be handed in before deadline 
for a chance to collect bonus points.
These assignments are graded sometime after deadline and at the latest before 
the first exam.
This depends on the work load of the tutor.
The correct solutions will be covered in detail during the first tutoring 
session after deadline, hence the inurgency to grade them quickly and if you 
miss the deadline you have missed the possibility.

Note that the bonus points are only valid during the first exam.

No tutoring is planned after the end of the course, i.e.\ after the last 
tutoring session scheduled in the course schedule.
If you are not done with your assignments during the course and want to be 
guaranteed tutoring you have to reregister for the next time the course is 
given.
Reregistration is a lower priority class of applicants for a course, all 
students applying for the course the first time have higher priority -- this 
includes students in reserve places too.

A final note, if you feel that you will not be done with the course on time, it 
is better to stop the course at an early stage.
If you register a break within three weeks of the course start, you will be in 
the higher priority class of applicants the next time you apply for the course.
You can register such a break yourself in the Student Portal.
Please contact the Student Office for further information regarding this early 
break from the course.


\printbibliography
\end{document}
