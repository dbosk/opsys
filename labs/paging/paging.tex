% $Id$
% Author:	Daniel Bosk <daniel.bosk@miun.se>
% Date:		5 Dec 2012
\documentclass[a4paper,nocourse]{miunasgn}
\usepackage[utf8]{inputenc}
\usepackage[swedish,english]{babel}
\usepackage[hyphens]{url}
\usepackage{hyperref}
\usepackage{prettyref,varioref}
\usepackage{color}
\usepackage{listings}
\usepackage{pdfpages}
\usepackage[nofancy,today]{svninfo}
\usepackage[natbib,style=numeric-comp,maxbibnames=99,sorting=none,backend=bibtex8]{biblatex}
\addbibresource{literature.bib}
\usepackage[varioref,prettyref,listings]{miunmisc}

\svnInfo $Id$

%\printanswers

\courseid{DT141G}
\course{Operating Systems}
\assignmenttype{Laboratory assignment}
\title{Paging Algorithms}
\author{Daniel Bosk\footnote{%
	This work is licensed under the Creative Commons Attribution-ShareAlike 3.0 
	Unported license.
	To view a copy of this license, visit 
	\url{http://creativecommons.org/licenses/by-sa/3.0/}.
}}
\date{\svnId}

\begin{document}
\maketitle
\thispagestyle{foot}
\tableofcontents


\section{Introduction}
\label{sec:Intro}
The paging algorithm employed in a system can have a huge impact on the 
performance of the system.
In this laboratory assignment you are to examine a some of the simplest page 
replacement algorithms using a simulator.


\section{Aim}
\label{sec:Aim}
The aim of this laboratory assignment is to further your understanding of the 
paging algorithms in question and the problems which have to be solved by this 
kind of algorithm.
It will also serve to examinate the following.
\begin{itemize}
		% $Id$
% Author:	Daniel Bosk <daniel.bosk@miun.se>
\item You can account for the fundamental function of the main logical 
components of an operating system, e.g.\ memory and process management, and 
explain their relations.
\item You understand the operating system interface against hardware, software, 
and users.
\item You are be able to explain the most common problems of resource 
allocation and synchronization, and be familiar with common solutions to these 
problems.
\item You can identify and understand the importance of some key parameters for 
performance in an operating system.

\end{itemize}


\section{Reading Instructions}
\label{sec:Reading}
% $Id$
This lecture gives an overview of the technicalities of the file system.
This is covered by Chapter 10 ``File System'' in \cite{Silberschatz2009osc}, or 
Chapter 11 ``File-System Interface'' in 
\cite{Silberschatz2013osc,Silberschatz2013intl}.



\section{Theory}
\label{sec:Theory}
The source code for the simulator is included with this assignment.
The simulator program-file is called \term{pager}, for the C version located in 
\term{cpager/}, or \term{pypager} for the Python version located in 
\term{pypager/}.

\subsection{The Simulator Written in C}

To use the C version you will have to compile it.
A \term{Makefile} is provided to make the process of compilation easier.
To compile the program, simply execute the command-line \term{make pager} in 
your terminal when your current working directory is \term{cpager/}.
When run, the simulator program reads a page reference string from standard 
input (stdin) and outputs its actions to standard output (stdout).

To switch between algorithms you should first make changes in one of the source 
files and then recompile the program.
To switch to a different algorithm, edit the file \term{pager.c} and change the 
argument to the function \code{mem_setalg}.
The algorithms that are supported by the simulator program can be found in the 
file \term{memalg.h}.
When done, recompile the program by running \term{make pager} in the terminal 
again.

You can also change the number of available pages and frames.
This is accomplished by changing the variables \code{npages} and \code{nframes} 
also in \term{pager.c}.

Running the paging simulator is rather straight-forward.
Simply execute it without arguments and it will wait for the page reference 
string on stdin.
The page reference string can be space or new-line separated (or both) and the 
output will be on the form of one action per line.
The page references can optionally be prefixed with either an \emph{r} or 
a \emph{w}; indicating that the frame should be read from or written to, 
respectively.
If no prefix is supplied, a read operation is assumed.
This means that you could use the reference strings from 
\cite{Silberschatz2009osc,Silberschatz2013osc,Silberschatz2013intl} directly, 
and these would be interpreted as reading operations.

To illustrate all this, the example in \prettyref{lst:reflength} illustrates 
how to easily count the number of page references in our reference string.
We see that we have 8 page references in our reference string.

\begin{terminal}[float,label={lst:reflength},caption={An example from the 
  UNIX-like terminal of how to compute the length of the reference string.}]
$ make pager
[lots of output]
$ cat ../fifo-refstr.txt
r1 w2 r3 r4
w5 w6 r2 r10
$ cat ../fifo-refstr.txt | wc -w
8
$
\end{terminal}

Listing \ref{lst:pager} illustrates the output of the simulator and how to 
easily count the number of page faults.
Of the 8 page references a total of 6 generated page faults.

\begin{terminal}[float,label={lst:pager},caption={An example showing a run of 
    the simulator.
    The last page reference is out of range and causes an abort trap to the 
    operating system, resulting in a killed process.
  }]
$ cat ../fifo-refstr.txt | ./pager
page 1 generated page fault
allocated free frame 0 to page 1
swapped in page 1
page 1 mapped to frame 0
page 2 generated page fault
allocated free frame 1 to page 2
swapped in page 2
page 2 mapped to frame 1
page 3 generated page fault
allocated free frame 2 to page 3
swapped in page 3
page 3 mapped to frame 2
page 4 generated page fault
allocated free frame 3 to page 4
swapped in page 4
page 4 mapped to frame 3
page 5 generated page fault
allocated free frame 4 to page 5
swapped in page 5
page 5 mapped to frame 4
page 6 generated page fault
swapped out page 1
swapped in page 6
page 6 mapped to frame 0
page 2 mapped to frame 1
page 10 generated abort trap: process killed
$ cat ../fifo-refstr.txt | ./pager | grep "page fault" | wc -l
6
\end{terminal}

\subsection{The Simulator Written in Python}

The simulator written in Python, named \term{pypager}, works in exactly the 
same manner as the one in C, the only difference is the programming language.
Since it is written in Python you do not have to compile it though.
Just go to the \term{pypager/} directory and run \term{./pypager} instead of 
\term{./pager} as in the instruction for the C version of the simulator.

To change the algorithms and number of frames or pages, edit the file 
\term{pager.py}.
Changing the algorithm used is done by changing the value of the variable 
\code{alg}.
There are lines prepared for this, just swtich to any of the other commented 
out lines in the code.
To change the pages or frames, modify the values of the variables \code{npages} 
or \code{nframes}, respectively.


\section{Work}
\label{sec:Work}
You should start by convincing yourself whether the provided replacement 
algorithms are correctly implemented.
Do this by testing different page reference strings and do the corresponding 
computations manually using the descriptions of the algorithms given by 
\citet[ch.\ 9.4]{Silberschatz2009osc,Silberschatz2013osc,Silberschatz2013intl}.

Next, you should find one page reference string for each algorithm which yields 
a minimum page-fault rate for that particular algorithm.
The trivial page reference string containing only one repeated page number is 
not allowed, all available pages must be referenced at least once.
Similarily you should find a page reference string which yields a maximum 
page-fault rate.

You may post your findings to the forum in the course platform if you avoid 
spoilers.
That is, you may post the \emph{length} of the page reference string and the 
number of page faults it generated -- but you may not post the string itself.

Finally, find proof of Belady's anomaly for the algorithms suffering from it.
%And, of course, correspondingly verify that this does not happen for the 
%algorithms not suffering from it.

The algorithms available are FIFO, Second-Chance, and Enhanced Second-Chance.


\section{Examination}
\label{sec:Examination}
Hand in a written report (in PDF format) describing your results, the 
conclusions you have drawn from these as well as why you drew these 
conclusions.
In essence, you should do the following:
\begin{questions}
	\question Answer whether the algorithms are correctly implemented or not?
	\begin{solution}
    ``Yes, after some extensive testing, I conclude that they are.''
		The students must provide convincing test scenarios.
	\end{solution}

	\question Provide a page reference string which generates a minimum 
	page-fault rate for each algorithm.
	\begin{solution}
    E.g.\ \code{0 1 2 3 4 5 6 7 8 9 9 9 9 9 9 ...} in the default 
    configuration.
    I.e.\ they get at most ten page faults (the initial page faults).
    They must also explain this, just a string of numbers is not sufficient to 
    pass.

    It is accomplished similarly for the other algorithms.
  \end{solution}

	\question Provide a page reference string which generates a maximum 
	page-fault rate for each algorithm.
	\begin{solution}
    E.g.\ \code{0 1 2 3 4 5 6 7 8 9 0 1 2 3 4 5 6 7 8 9 ...} in the default 
    configuration.
    I.e.\ every page reference generates a page fault.
    They must also explain this, just a string of numbers is not sufficient to 
    pass.

    It is accomplished similarly for the other algorithms.
	\end{solution}

	\question Provide evidence of Belady's anomaly for the algorithms suffering 
	from it.
	\begin{solution}
    Using the page reference string \code{1 2 3 4 1 2 5 1 2 3 4 5}, with FIFO 
    configured with three frames (the number of pages does not matter), we get 
    nine page faults.
		Increasing the number of frames to four, which should decrease the number 
		of page faults as we have more frames, actually increases the number of 
		page faults to ten.
		This is proof of Belady's anomaly.

    It is accomplished similarly for the other algorithms.
	\end{solution}
\end{questions}
You should, as always, provide convincing arguments for your answers and use 
correct references where applicable.

To make life easy for you a report template is attached, see \emph{report.tex}.
(This file is also included in \prettyref{app:report}.)
You may produce your report using any tool, but it should look the same as the 
one generated by the template.
If you use it you can simply execute \term{make report.pdf} in the terminal to 
compile the PDF-file.


\printbibliography


\appendix
\section{A Report Template}
\label{app:report}
Here follows the inclusion of a LaTeX report template.
This file is also included in the sources as \term{report.tex}.

\lstinputlisting[style=code,language=tex]{report.tex}

Here follows a compiled version of the above code.

\includepdf[pages=-]{report.pdf}

\end{document}
