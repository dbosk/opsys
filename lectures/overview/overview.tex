% $Id$
% Author:	Daniel Bosk <daniel.bosk@miun.se>
%\documentclass[handout]{beamer}
\documentclass{beamer}
\usepackage[utf8]{inputenc}
\usepackage[T1]{fontenc}
\usepackage[swedish,english]{babel}
\usepackage{url}
\usepackage{graphicx}
\usepackage[today,nofancy]{svninfo}
\usepackage[natbib,style=alphabetic,maxbibnames=99,backend=bibtex8]{biblatex}
\addbibresource{literature.bib}

\svnInfo $Id$

\mode<presentation>
{
  \usetheme{Frankfurt}
  \setbeamercovered{transparent}
  \usecolortheme{seagull}
}
%\def\newblock{\hskip .11em plus .33em minus .07em}
\setbeamertemplate{footline}{\insertframenumber}

\title{%
  System Organisation and Structure
}
\author{Daniel Bosk\footnote{%
	\tiny
  This work is licensed under the Creative Commons Attribution-ShareAlike 3.0 
  Unported license.
	To view a copy of this license, visit 
	\url{http://creativecommons.org/licenses/by-sa/3.0/}.
}}
\institute[MIUN ICS]{%
  Department of Information and Communication Systems (ICS),\\
  Mid Sweden University, Sundsvall.
}
\date{\svnId}


\AtBeginSection[]{%
  \begin{frame}<beamer>{Overview}
		\tableofcontents[currentsection]
	\end{frame}
}

\begin{document}

\begin{frame}
  \titlepage
\end{frame}

\begin{frame}{Overview}
	\tableofcontents
	% You might wish to add the option [pausesections]
\end{frame}


% Since this a solution template for a generic talk, very little can
% be said about how it should be structured. However, the talk length
% of between 15min and 45min and the theme suggest that you stick to
% the following rules:  

% - Exactly two or three sections (other than the summary).
% - At *most* three subsections per section.
% - Talk about 30s to 2min per frame. So there should be between about
%   15 and 30 frames, all told.


\section{Computer Systems}

\subsection{Overview of the System Structure}

\begin{frame}{\insertsubsectionhead}
  \begin{figure}
    \includegraphics[width=\textwidth]{syscomponents.pdf}
    \caption{Abstract view of the components of a computer system.
      Image: \cite[p.\ 4]{Silberschatz2009osc}.
    }
  \end{figure}
\end{frame}

%\begin{frame}{\insertsubsectionhead}
%  \begin{itemize}
%    \item Hardware:
%      \begin{itemize}
%        \item CPU,
%        \item memory,
%        \item I/O-devices.
%      \end{itemize}
%    \item Software:
%  \end{itemize}
%\end{frame}

\begin{frame}{\insertsubsectionhead}{User View}
  \begin{itemize}
    \item Ease of use of a system.
    \item Resource utilization.
    \item Can be on different systems: workstation, server, mainframe.
  \end{itemize}
\end{frame}

\begin{frame}{\insertsubsectionhead}{System View}
  \begin{itemize}
    \item Resource allocator.
    \item Control program.
  \end{itemize}
\end{frame}

\begin{frame}{\insertsubsectionhead}
  \begin{figure}
    \includegraphics[width=\textwidth]{services.pdf}
    \caption{A detailed view of the operating system structure.
      Image: \cite[p.\ 50]{Silberschatz2009osc}.
    }
  \end{figure}
\end{frame}

\begin{frame}{\insertsubsectionhead}
  \begin{figure}
    \includegraphics[width=\textwidth]{sysstruct.pdf}
    \caption{A systematic overview of a computer system.
      Image: \cite[p.\ 8]{Silberschatz2009osc}.
    }
  \end{figure}
\end{frame}

\begin{frame}{\insertsubsectionhead}
  \begin{itemize}
    \item Bootstrap program.
    \item Interrupts.
    \item System calls.
  \end{itemize}
\end{frame}

\begin{frame}{\insertsubsectionhead}
  \begin{figure}
    \includegraphics[width=\textwidth]{interrupt.pdf}
    \caption{Timeline for interrupt processing and process execution.
      Image: \cite[p.\ 9]{Silberschatz2009osc}.
    }
  \end{figure}
\end{frame}

\subsection{Storage}

\begin{frame}{\insertsubsectionhead}
  \begin{itemize}
    \item Random-access memory.
    \item von Neumann architecture.
    \item Instruction registers.
  \end{itemize}
\end{frame}

\begin{frame}{\insertsubsectionhead}
  \begin{itemize}
    \item Primary and secondary storage.
    \item Volatile and non-volatile storage.
  \end{itemize}
\end{frame}

\begin{frame}{\insertsubsectionhead}
  \begin{figure}
    \includegraphics[height=0.8\textheight]{storagehier.pdf}
    \caption{Storage-device hierarchy.
      Image: \cite[p.\ 11]{Silberschatz2009osc}.
    }
  \end{figure}
\end{frame}

\subsection{I/O Structure}

\begin{frame}{\insertsubsectionhead}
  \begin{itemize}
    \item Controllers.
    \item Device driver.
    \item Direct-memory access (DMA).
  \end{itemize}
\end{frame}

\subsection{Architecture}

\begin{frame}{\insertsubsectionhead}
  \begin{figure}
    \includegraphics[height=0.8\textheight]{arch.pdf}
    \caption{Overview of computer architecture.
      Image: \cite[p. 13]{Silberschatz2009osc}.
    }
  \end{figure}
\end{frame}

\begin{frame}{\insertsubsectionhead}
  \begin{figure}
    \includegraphics[width=\textwidth]{smp.pdf}
    \caption{Symmetric multiprocessing architecture (SMP).
      Image: \cite[p.\ 15]{Silberschatz2009osc}.
    }
  \end{figure}
\end{frame}

\begin{frame}{\insertsubsectionhead}
  \begin{figure}
    \includegraphics[width=\textwidth]{cluster.pdf}
    \caption{A clustered system.
      Image: \cite[p.\ 18]{Silberschatz2009osc}.
    }
  \end{figure}
\end{frame}

\subsection{OS Structure}

\begin{frame}{\insertsubsectionhead}
  \begin{itemize}
    \item Multiprogramming.
    \item Time-sharing, multitasking.
    \item Interactive computer system.
    \item Batch processing.
  \end{itemize}
\end{frame}

\begin{frame}{\insertsubsectionhead}
  \begin{figure}
    \includegraphics[height=0.8\textheight]{multiprog.pdf}
    \caption{Memory layout for a multiprogramming system.
      Image: \cite[p.\ 19]{Silberschatz2009osc}.
    }
  \end{figure}
\end{frame}

\begin{frame}{\insertsubsectionhead}
  \begin{itemize}
    \item User and kernel mode.
    \item Mode bit.
    \item Privileged instructions.
  \end{itemize}
\end{frame}

\begin{frame}{\insertsubsectionhead}
  \begin{figure}
    \includegraphics[width=\textwidth]{dualmode.pdf}
    \caption{Example of transition from user to kernel mode.
      Image: \cite[p.\ 21]{Silberschatz2009osc}.
    }
  \end{figure}
\end{frame}

\subsection{Caching}

\begin{frame}{\insertsubsectionhead}
  \begin{figure}
    \includegraphics[width=\textwidth]{cache.pdf}
    \caption{Transfer of \(A\) from disk to register.
      Image: \cite[p.\ 28]{Silberschatz2009osc}.
    }
  \end{figure}
\end{frame}

\subsection{Virtual machines}

\begin{frame}{\insertsubsectionhead}
  \begin{figure}
    \includegraphics[height=0.8\textheight]{vm.pdf}
    \caption{An illustration of virtual machines compared to non-virtual 
      machines.
      Image: \cite{Silberschatz2013intl}.
    }
  \end{figure}
\end{frame}


%%%%%%%%%%%%%%%%%%%%%%

\begin{frame}[allowframebreaks]{Referenser}
  \small
  \printbibliography
\end{frame}

\end{document}

