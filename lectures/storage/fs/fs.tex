% $Id$
% Author:	Daniel Bosk <daniel.bosk@miun.se>
%\documentclass[handout]{beamer}
\documentclass{beamer}
\usepackage[utf8]{inputenc}
\usepackage[T1]{fontenc}
\usepackage[swedish,english]{babel}
\usepackage{url}
\usepackage{varioref,prettyref}
\usepackage{graphicx}
\usepackage{listings}
\usepackage[today,nofancy]{svninfo}
\usepackage[natbib,style=alphabetic,maxbibnames=99,backend=bibtex8]{biblatex}
\addbibresource{literature.bib}
\usepackage[listings,varioref,prettyref]{miunmisc}

\svnInfo $Id$

\mode<presentation>
{
  \usetheme{Frankfurt}
  \setbeamercovered{transparent}
  \usecolortheme{seagull}
}
%\def\newblock{\hskip .11em plus .33em minus .07em}
\setbeamertemplate{footline}{\insertframenumber}

\title{%
  Implementing File System
}
\author{Daniel Bosk\footnote{%
	\tiny
  This work is licensed under the Creative Commons Attribution-ShareAlike 3.0 
  Unported license.
	To view a copy of this license, visit 
	\url{http://creativecommons.org/licenses/by-sa/3.0/}.
}}
\institute[MIUN ICS]{%
  Department of Information and Communication Systems (ICS),\\
  Mid Sweden University, Sundsvall.
}
\date{\svnId}


\AtBeginSection[]{%
  \begin{frame}<beamer>{Overview}
		\tableofcontents[currentsection]
	\end{frame}
}

\begin{document}

\begin{frame}
  \titlepage
\end{frame}

\begin{frame}{Overview}
	\tableofcontents
	% You might wish to add the option [pausesections]
\end{frame}


% Since this a solution template for a generic talk, very little can
% be said about how it should be structured. However, the talk length
% of between 15min and 45min and the theme suggest that you stick to
% the following rules:  

% - Exactly two or three sections (other than the summary).
% - At *most* three subsections per section.
% - Talk about 30s to 2min per frame. So there should be between about
%   15 and 30 frames, all told.

\section{File Systems}

\subsection{File-System Structure}

\begin{frame}{\insertsubsectionhead}
  \begin{itemize}
    \item The disk stores data in blocks, normally 512 to 4096 bytes.

    \item FS provides an easy and convenient way of accessing this data.

    \item The first problem is how the FS should look to the user.

    \item The second problem is creating algorithms and data structures to map 
      the logical file system (what the user sees) onto the physical secondary 
      storage devices.

    \item We will now focus on the second problem.

  \end{itemize}
\end{frame}

\begin{frame}{\insertsubsectionhead}
  \begin{figure}
    \includegraphics[height=0.7\textheight]{fshier.pdf}
    \caption{Layered hierarchy of file system structure.
      Image: \cite{Silberschatz2013osc}.}
  \end{figure}
\end{frame}

\begin{frame}{\insertsubsectionhead}{I/O Control}
  \begin{itemize}
    \item This layer consists of device drivers and interrupt handlers.

    \item This layer transfers information between memory and disk.

    \item From upper layers: fetch block 123.

    \item To lower layers: hardware specific instructions.

    \item It ususally writes specific bit patterns (magic numbers) to special 
      locations in the I/O controller's memory.

  \end{itemize}
\end{frame}

\begin{frame}{\insertsubsectionhead}{Basic File System}
  \begin{itemize}
    \item The basic file system issues generic commands to the appropriate 
      device driver; read from block \(X\), write to block \(Y\).

    \item Each physcial block is identified by its numeric disk address; e.g.\ 
      drive 1, cylinder 73, track 2, sector 10.

    \item This layer also manages buffers and caches that hold various 
      directory and data blocks.

    \item The manager allocated a block in the buffer before transfer can 
      occur.
      Thus is needs to keep track of free places in the buffer, and free space 
      when there is none.

    \item The caches are to speed up system performance, these must be kept 
      up-to-date to make this work.

  \end{itemize}
\end{frame}

\begin{frame}{\insertsubsectionhead}{File-Organisation Module}
  \begin{itemize}
    \item This part of the FS keeps track of actual files and maps these to 
      their blocks in the physical device.

    \item The file-organisation module translates logical block addresses to 
      physical block addresses for the basic file system to transfer.

    \item This layer also keeps track of free physical blocks in the device.

  \end{itemize}
\end{frame}

\begin{frame}{\insertsubsectionhead}{Logical File System}
  \begin{itemize}
    \item This layer manages meta-data information, i.e.\ everything except the 
      actual data (file contents).

    \item The logical file system manages the directory structure.

    \item It converts symbolic file names into the IDs the file-organisation 
      module needs.

    \item It maintains this information in a file-control block (FCB), or 
      inode.

    \item This layer is also responsible for protection and security, so it has 
      access permissions for all files.

  \end{itemize}
\end{frame}

\subsection{Volumes}

\begin{frame}{\insertsubsectionhead}
  \begin{itemize}
    \item Each volume contains a boot control block and a volume control block.

    \item In UFS the boot control block is called the boot block and the volume 
      control block is called a superblock.

    \item In NTFS it's called the partition boot sector.
      The volume control block is called the master file table (MFT).

    \item A directory structure is used to organise the files: in UFS, these 
      include file names and inode numbers;
      in NTFS this is stored in the MFT.

    \item Finally, a per file FCB has all information about a file, it contains 
      a unique identifier to match directory entries.
      (In NTFS this is in the MFT.)

  \end{itemize}
\end{frame}

\begin{frame}{\insertsubsectionhead}
  \begin{itemize}
    \item Data about the FS is kept in memory via caches.
    \item This data is loaded on mount time.
    \item It's updated during operation.
    \item It's discarded on dismount.
  \end{itemize}
\end{frame}

\subsection{Kernel Data-Structures}

\begin{frame}{\insertsubsectionhead}
  \begin{itemize}
    \item The OS keeps an in-memory mount table about each mounted volume.
      (Look at mount(8).)

    \item An in-memory directory-structure cache holds the directory 
      information about recently accessed directories.
      (Try \term{time find /some/subdir -print > /dev/null} two times in 
      a row.)

    \item The system-wide open-file table keeps a copy of the FCB of each open 
      file.
      (Look at lsof(8).)

    \item The per-process open-file table keeps a pointer to the appropriate 
      entry in the system-wide open-file table.

    \item Buffers hold file-system blocks when they are being read from or 
      written to disk.

  \end{itemize}
\end{frame}

\begin{frame}[fragile]{\insertsubsectionhead}
  \begin{terminal}
\$ time find ~ -print > /dev/null
real    0m47.535s
user    0m0.479s
sys     0m1.779s
\$
  \end{terminal}
  \begin{terminal}
\$ time find ~ -print > /dev/null
real    0m0.180s
user    0m0.080s
sys     0m0.100s
\$
  \end{terminal}
\end{frame}

\begin{frame}{\insertsubsectionhead}
  \begin{figure}
    \includegraphics[height=0.7\textheight]{fcb.pdf}
    \caption{A typical file-control block.
      Image: \cite{Silberschatz2013osc}.}
  \end{figure}
\end{frame}

\subsection{Operations on Files}

\begin{frame}{\insertsubsectionhead}
  \begin{itemize}
    \item A process which wants to create a new file make a request from the 
      logical file system.

    \item The logical file sytem allocates an FCB, either creates a new or take 
      one of the available in case the FCB:s are created at FS creation.

    \item The open(2) system call passes a symbolic name to the logical file 
      system.

    \item This first searches the system-wide open-file table, if found it 
      points an entry in the process's open-file table there.

    \item If not found, the directory structure is searched, when found the FCB 
      is copied into a new entry in the system-wide open-file table.

    \item We must also track how many processes keep the file open.

  \end{itemize}
\end{frame}

\begin{frame}{\insertsubsectionhead}
  \begin{figure}
    \includegraphics[height=0.7\textheight]{openread.pdf}
    \caption{In-memory file-system structures; (a) file open, (b) file read.
      Image: \cite{Silberschatz2013osc}.}
  \end{figure}
\end{frame}


\section{Virtual File Systems}

\subsection{File System Layeres}

\begin{frame}{\insertsubsectionhead}
  \begin{figure}
    \includegraphics[height=0.7\textheight]{vfs.pdf}
    \caption{A schematic view of the virtual file system layer.
      Image: \cite{Silberschatz2013osc}.}
  \end{figure}
\end{frame}

\begin{frame}{\insertsubsectionhead}
  \begin{itemize}
    \item The virtual file system (VFS) separates generic operations from their 
      implementation.
      I.e.\ each FS implements the VFS interface.

    \item The the OS's FS operations just use the VFS interface, no need to 
      bother about what's underneath.

    \item This way we can even implement remote file systems like NFS.

  \end{itemize}
\end{frame}

\begin{frame}{\insertsubsectionhead}
  \begin{figure}
    \includegraphics[height=0.7\textheight]{nfs.pdf}
    \caption{A schematic of VFS with NFS.
      Image: \cite{Silberschatz2013osc}.}
  \end{figure}
\end{frame}


\section[Block Allocation]{Block Allocation and Free Block Algorithms}

\subsection{Directories}

\begin{frame}{\insertsubsectionhead}
  \begin{itemize}
    \item We need to store and represent our directoy structure in some way.

    \item We have to easy alternatives: linear lists and hash tables.
  \end{itemize}
\end{frame}

\begin{frame}{\insertsubsectionhead}
  \begin{itemize}
    \item The linear list is simple.

    \item We simply store the file information in a list, from beginning to 
      end.
      
    \item To create a file we go through the list to see there's no existing 
      file by that name yet, then we create a new entry.

    \item To remove a file we free its allocated entry, e.g.\ by setting it to 
      null or replacing it with the last directory entry and reduce the length 
      of the directory.

  \end{itemize}
\end{frame}

\begin{frame}{\insertsubsectionhead}
  \begin{itemize}
    \item This is computationally bad, since searching through a list takes 
      time.

    \item It's easier if it's sorted, then the search is faster.
      Problem is to keep it sorted.

    \item We can use a hash table instead, hash function computes a hash value 
      from the symbolic name.

    \item Problem is collisions, if the collisions are few and evenly spread we 
      can use a linked list for this.

  \end{itemize}
\end{frame}

\subsection{Contiguous Allocation}

\begin{frame}{\insertsubsectionhead}
  \begin{figure}
    \includegraphics[height=0.7\textheight]{contiguous.pdf}
    \caption{An example of contiguous allocation of file data.
      Image: \cite{Silberschatz2013osc}.}
  \end{figure}
\end{frame}

\subsection{Linked Allocation}

\begin{frame}{\insertsubsectionhead}
  \begin{figure}
    \includegraphics[height=0.7\textheight]{linked.pdf}
    \caption{An example of linked allocation of file data.
      Image: \cite{Silberschatz2013osc}.}
  \end{figure}
\end{frame}

\begin{frame}{\insertsubsectionhead}
  \begin{figure}
    \includegraphics[height=0.7\textheight]{fat.pdf}
    \caption{An example of a file-allocation table (FAT).
      Image: \cite{Silberschatz2013osc}.}
  \end{figure}
\end{frame}

\subsection{Indexed Allocation}

\begin{frame}{\insertsubsectionhead}
  \begin{figure}
    \includegraphics[height=0.7\textheight]{indexed.pdf}
    \caption{An example of indexed allocation.
      Image: \cite{Silberschatz2013osc}.}
  \end{figure}
\end{frame}

\begin{frame}{\insertsubsectionhead}
  \begin{figure}
    \includegraphics[height=0.7\textheight]{inode.pdf}
    \caption{An example of the UFS inode.
      Image: \cite{Silberschatz2013osc}.}
  \end{figure}
\end{frame}



%%%%%%%%%%%%%%%%%%%%%%

\begin{frame}[allowframebreaks]{Referenser}
  \small
  \printbibliography
\end{frame}

\end{document}

