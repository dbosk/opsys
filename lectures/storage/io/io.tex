% $Id$
% Author:	Daniel Bosk <daniel.bosk@miun.se>
%\documentclass[handout]{beamer}
\documentclass{beamer}
\usepackage[utf8]{inputenc}
\usepackage[T1]{fontenc}
\usepackage[english,swedish]{babel}
\usepackage{url}
\usepackage{varioref,prettyref}
\usepackage{graphicx}
\usepackage[today,nofancy]{svninfo}
\usepackage[natbib,style=alphabetic,maxbibnames=99,backend=bibtex8]{biblatex}
\addbibresource{literature.bib}
\usepackage[varioref,prettyref]{miunmisc}

\svnInfo $Id$

\mode<presentation>
{
  \usetheme{Frankfurt}
  \setbeamercovered{transparent}
  \usecolortheme{seagull}
}
%\def\newblock{\hskip .11em plus .33em minus .07em}
\setbeamertemplate{footline}{\insertframenumber}

\title{%
  I/O Systems
}
\author{Daniel Bosk\footnote{%
	\tiny
  This work is licensed under the Creative Commons Attribution-ShareAlike 3.0 
  Unported license.
	To view a copy of this license, visit 
	\url{http://creativecommons.org/licenses/by-sa/3.0/}.
}}
\institute[MIUN ICS]{%
  Department of Information and Communication Systems (ICS),\\
  Mid Sweden University, Sundsvall.
}
\date{\svnId}


\AtBeginSection[]{%
  \begin{frame}<beamer>{Overview}
		\tableofcontents[currentsection]
	\end{frame}
}

\begin{document}

\begin{frame}
  \titlepage
\end{frame}

\begin{frame}{Overview}
	\tableofcontents
	% You might wish to add the option [pausesections]
\end{frame}

\begin{frame}{Literature}
  % $Id$
This lecture gives an overview of the technicalities of the file system.
This is covered by Chapter 10 ``File System'' in \cite{Silberschatz2009osc}, or 
Chapter 11 ``File-System Interface'' in 
\cite{Silberschatz2013osc,Silberschatz2013intl}.

\end{frame}


% Since this a solution template for a generic talk, very little can
% be said about how it should be structured. However, the talk length
% of between 15min and 45min and the theme suggest that you stick to
% the following rules:  

% - Exactly two or three sections (other than the summary).
% - At *most* three subsections per section.
% - Talk about 30s to 2min per frame. So there should be between about
%   15 and 30 frames, all told.


\section{Overview}

\subsection{Hardware}

\begin{frame}{\insertsubsectionhead}
  \begin{figure}
    \includegraphics[height=0.7\textheight]{hw.pdf}
    \caption{An overview of computer hardware.
      Image: \cite{Silberschatz2013osc}.}
  \end{figure}
\end{frame}

\begin{frame}{\insertsubsectionhead}
  \begin{itemize}
    \item A device communicates with the machine via a port.

    \item If several devices share a common set of wires for communication it's 
      called a bus.

    \item A bus is a set of wires and a well-defined protocol.

    \item A controller is a set of electronics which can operate a port, bus, 
      or a device.

  \end{itemize}
\end{frame}

\begin{frame}{\insertsubsectionhead}
  \begin{itemize}
    \item The controller has a set of registers to which the CPU can write and 
      from which it can also read.

    \item These can be read and written using the bus protocol.

    \item These can also be mapped into certain memory regions and be used for 
      memory-mapped I/O.

    \item An I/O port typically consists of four registers: status, control, 
      data-in, data-out.

  \end{itemize}
\end{frame}

\subsection{Polling or Interrupts}

\begin{frame}{\insertsubsectionhead}
  \begin{itemize}
  \end{itemize}
\end{frame}


%%%%%%%%%%%%%%%%%%%%%%

\begin{frame}[allowframebreaks]{Referenser}
  \small
  \printbibliography
\end{frame}

\end{document}

