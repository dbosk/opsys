% $Id$
% Author:	Daniel Bosk <daniel.bosk@miun.se>
%\documentclass[handout]{beamer}
\documentclass{beamer}
\usepackage[utf8]{inputenc}
\usepackage[T1]{fontenc}
\usepackage[swedish,english]{babel}
\usepackage{url}
\usepackage{varioref,prettyref}
\usepackage{graphicx}
\usepackage{multicol}
\usepackage{listings}
\usepackage[today,nofancy]{svninfo}
\usepackage[natbib,style=alphabetic,maxbibnames=99,backend=bibtex8]{biblatex}
\addbibresource{literature.bib}
\usepackage[listings,varioref,prettyref]{miunmisc}

\svnInfo $Id$

\mode<presentation>
{
  \usetheme{Frankfurt}
  \setbeamercovered{transparent}
  \usecolortheme{seagull}
}
%\def\newblock{\hskip .11em plus .33em minus .07em}
\setbeamertemplate{footline}{\insertframenumber}

\title{%
  File Systems
}
\author{Daniel Bosk\footnote{%
	\tiny
  This work is licensed under the Creative Commons Attribution-ShareAlike 3.0 
  Unported license.
	To view a copy of this license, visit 
	\url{http://creativecommons.org/licenses/by-sa/3.0/}.
}}
\institute[MIUN ICS]{%
  Department of Information and Communication Systems (ICS),\\
  Mid Sweden University, Sundsvall.
}
\date{\svnId}


\AtBeginSection[]{%
  \begin{frame}<beamer>{Overview}
    \begin{multicols}{2}
      \tableofcontents[currentsection]
    \end{multicols}
	\end{frame}
}

\begin{document}

\begin{frame}
  \titlepage
\end{frame}

\begin{frame}{Overview}
  \begin{multicols}{2}
    \tableofcontents
    % You might wish to add the option [pausesections]
  \end{multicols}
\end{frame}


% Since this a solution template for a generic talk, very little can
% be said about how it should be structured. However, the talk length
% of between 15min and 45min and the theme suggest that you stick to
% the following rules:  

% - Exactly two or three sections (other than the summary).
% - At *most* three subsections per section.
% - Talk about 30s to 2min per frame. So there should be between about
%   15 and 30 frames, all told.


\section{File Concept}

\subsection{File Meta-Data and Operations}

\begin{frame}{\insertsubsectionhead}
  \begin{itemize}
    \item A file is the smallest logical unit in the file system.

    \item Its contents is defined by its creator, hence the data inside can 
      have any structure.

    \item From the perspective of the OS there is no difference between to 
      files other than what can be seen in the meta-data.

  \end{itemize}
\end{frame}

\begin{frame}{\insertsubsectionhead}{Attributes}
  \begin{description}
    \item[Name] The symbolic file name for humans.

    \item[Identifier] A unique number identifying the file within the file 
      system (FS).

    \item[Location] The file must be stored on some device.

    \item[Size] The file has a size which depends on how much data is in it, 
      and how much is needed for storage.

    \item[Protection] Access control information must also be available.

    \item[Time] Date and time, for audit recording and other useful things.

  \end{description}
\end{frame}

\begin{frame}{\insertsubsectionhead}{Operations}
  \begin{description}
    \item[Creating] One must be able to create a new file, i.e.\ allocate space 
      and entry for the file.

    \item[Writing] We need some support from the OS for writing to a file, 
      a system call.

    \item[Reading] We also need a system call for reading from a file.
      This must keep track of where in the file we currently are, 
      current-file-position pointer.

    \item[Repositioning] We must be able to move within the file.

    \item[Delete] And we must be able to remove the file if it's no longer 
      needed.

  \end{description}
\end{frame}

\begin{frame}{\insertsubsectionhead}
  \begin{itemize}
    \item We need some other system calls and structures too.

    \item The OS uses an open(2) and a close(2) system call.

    \item OS keeps track of open files using the open-file table.
      Can be seen using lsof(1).

    \item The OS also has a per-process open-file table, this is where stdin 
      and stdout resides.

    \item Keeps a file pointer, file-open count, disk location of the file.

  \end{itemize}
\end{frame}

\begin{frame}[fragile]{\insertsubsectionhead}
  \begin{terminal}[basicstyle=\tiny\tt,numberstyle=\tiny]
\$ lsof .
COMMAND   PID   USER   FD   TYPE DEVICE SIZE/OFF    NODE NAME
bash     6729 danbos  cwd    DIR    8,5     4096 8667228 .
vim     13268 danbos  cwd    DIR    8,5     4096 8667228 .
bash    32564 danbos  cwd    DIR    8,5     4096 8667228 .
lsof    32565 danbos  cwd    DIR    8,5     4096 8667228 .
lsof    32566 danbos  cwd    DIR    8,5     4096 8667228 .
\$
  \end{terminal}
\end{frame}

\subsection{Locks on Files}

\begin{frame}{\insertsubsectionhead}
  \begin{itemize}
    \item What happens if two processes want to open the same file?

    \item We have shared and exclusive locks.

    \item OS also provides either mandatory or advisory file-locking 
      mechanisms.

    \item Windows has mandatory, UNIX has advisory.

  \end{itemize}
\end{frame}

\subsection{File Types and Internal Structure}

\begin{frame}{\insertsubsectionhead}
  \begin{itemize}
    \item The file name is for the user.
      It can have an extension as a hint to the user and OS about its type.

    \item To see what type a file really is, try file(1).

    \item MacOS X keeps track of which program created the file, to use that 
      for later opening again.

    \item Windows is entirely extension based.

    \item UNIX-like systems depends.
      It can use a magic number inside, a hint from the extension, or the 
      file(1) utility.

  \end{itemize}
\end{frame}

\begin{frame}{\insertsubsectionhead}
  \begin{itemize}
    \item From the operating system point of view, a file is simply a stream of 
      bytes.

    \item The OS packs a certain number of logical records into a physical 
      block on disk.
      Usually a logical record is a byte.

    \item Disk space is allocated in physical blocks on the device, usually 
      somewhere between 512 to 4096 bytes.

    \item If a file is less than the block size we have internal fragmentation.

  \end{itemize}
\end{frame}


\section{Access Methods}

\subsection{Sequential Access}

\begin{frame}{\insertsubsectionhead}
  \begin{itemize}
    \item What would you say was the biggest difference between the casette 
      band and the CD?

    \item The casette band is a good example of sequential access, the CD is 
      more close to direct access.

    \item You start reading from the beginning and can read forward or rewind.

    \item This simple model of access also works on random-access devices.

  \end{itemize}
\end{frame}

\subsection{Direct Access}

\begin{frame}{\insertsubsectionhead}
  \begin{itemize}
    \item In the direct-access model the file is made up of fixed-length 
      logical records.

    \item These records can be accessed in no particular order.

    \item The system doesn't need to search for the record by winding, it can 
      just compute the location and go there directly.

  \end{itemize}
\end{frame}


\section{Directory and Disk structure}

\subsection{Partitions and Volumes}

\begin{frame}{\insertsubsectionhead}
  \begin{itemize}
    \item A storage device can be partitioned into several partitions.
    \item Then one can fit several file systems in one device.
    \item One can even leave space for swap space or raw disk.
    \item Partitions can also be called slices or minidisks.
    \item A file system can be created in any of these parts.
    \item A partition containing a file system is called a volume.
    \item The volume may be part of a disk, a whole disk or several disks in 
      a RAID formation.
    \item Each volume has a device directory or volume table of contents to 
      keep track of all files.
  \end{itemize}
\end{frame}

\begin{frame}{\insertsubsectionhead}
  \begin{itemize}
    \item There are many types of file systems!
    \item Memory-based file sytems, various types of virtual file systems.
  \end{itemize}
\end{frame}

\subsection{Directory Overview}

\begin{frame}{\insertsubsectionhead}
  \begin{itemize}
    \item The directory can be seen as a symbol table translating names into 
      directory entries.

    \item In a directory we can search for files, create and delete files, 
      rename a file, traverse to subdirectories or parent directories.

  \end{itemize}
\end{frame}

\subsection{Single-Level and Two-Level Directory}

\begin{frame}{\insertsubsectionhead}
  \begin{figure}
    \includegraphics[width=\textwidth]{onelvldir.pdf}
    \caption{A one-level directory structure.
      Image: \cite{Silberschatz2013osc}.}
  \end{figure}
\end{frame}

\begin{frame}{\insertsubsectionhead}
  \begin{figure}
    \includegraphics[width=\textwidth]{twolvldir.pdf}
    \caption{A two-level directory structure.
      Image: \cite{Silberschatz2013osc}.}
  \end{figure}
\end{frame}

\subsection{Tree-Structured Directories}

\begin{frame}{\insertsubsectionhead}
  \begin{figure}
    \includegraphics[height=0.7\textheight]{treedir.pdf}
    \caption{A tree-structured file system.
      Image: \cite{Silberschatz2013osc}.}
  \end{figure}
\end{frame}

\subsection{Graph Directories}

\begin{frame}{\insertsubsectionhead}
  \begin{figure}
    \includegraphics[height=0.7\textheight]{acyclicdir.pdf}
    \caption{An acyclic graph directory structure.
      Image: \cite{Silberschatz2013osc}.}
  \end{figure}
\end{frame}

\begin{frame}{\insertsubsectionhead}
  \begin{figure}
    \includegraphics[height=0.7\textheight]{graphdir.pdf}
    \caption{A general graph directory structure.
      Image: \cite{Silberschatz2013osc}.}
  \end{figure}
\end{frame}

\begin{frame}{\insertsubsectionhead}
  \begin{itemize}
    \item Directories can share subdirectories.
    \item Can be implemented using links.
  \end{itemize}
\end{frame}


\section{File System Mounting}

\subsection{Mount Points}

\begin{frame}{\insertsubsectionhead}
  \begin{itemize}
    \item FS must be mounted, just as a file must be opened.

    \item An FS can be mounted in any directory using a special directory entry 
      pointing to the root directory of the mounted file system.

    \item Windows uses an extended two-level directory structure with devices 
      and volumes assigned drive letters.
      Then volumes have a general graph directory structure.
  \end{itemize}
\end{frame}

\begin{frame}{\insertsubsectionhead}
  \begin{figure}
    \includegraphics[height=0.7\textheight]{unmounted.pdf}
    \caption{Two file systems, one is unmounted.
      Image: \cite{Silberschatz2013osc}.}
  \end{figure}
\end{frame}

\begin{frame}{\insertsubsectionhead}
  \begin{figure}
    \includegraphics[height=0.7\textheight]{mounted.pdf}
    \caption{The previously unmounted file system is now mounted.
      Note that the subdirectories are invisible when the directory is used as 
      a mount point.
      Image: \cite{Silberschatz2013osc}.}
  \end{figure}
\end{frame}

\begin{frame}{\insertsubsectionhead}
  \begin{itemize}
    \item In UNIX-like systems the command mount(8) is used to work with 
      mounting volumes.

    \item Without arguments mount(8) will show information on all currently 
      mounted volumes in the system.

  \end{itemize}
\end{frame}

\subsection{Remote File Systems}

\begin{frame}{\insertsubsectionhead}
  \begin{itemize}
    \item These FSs are mounted as other FSs, although they need some special 
      care in the kernel.
  \end{itemize}
\end{frame}


\section{Protection}

\subsection{Access Control}

\begin{frame}[fragile]{\insertsubsectionhead}
  \begin{terminal}[basicstyle=\tiny\tt,numberstyle=\tiny]
\$ ls -lha
total 60K
drwxr-x--- 2 danbos danbos 4.0K Dec  3 13:16 .
drwxr-x--- 6 danbos danbos 4.0K Dec  3 10:07 ..
-rw-r----- 1 danbos danbos  11K Dec  3 13:16 fs.tex
-rw-r----- 1 danbos danbos  28K Dec  3 13:16 .fs.tex.swp
-rw-r----- 1 danbos danbos  100 Dec  3 10:12 literature.tex
-rw-r----- 1 danbos danbos  551 Dec  3 10:10 Makefile
\$
  \end{terminal}
\end{frame}


%%%%%%%%%%%%%%%%%%%%%%

\begin{frame}[allowframebreaks]{Referenser}
  \small
  \printbibliography
\end{frame}

\end{document}

