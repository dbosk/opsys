% $Id$
% Author:	Daniel Bosk <daniel.bosk@miun.se>
%\documentclass[handout]{beamer}
\documentclass{beamer}
\usepackage[utf8]{inputenc}
\usepackage[T1]{fontenc}
\usepackage[swedish,english]{babel}
\usepackage{url}
\usepackage{varioref,prettyref}
\usepackage{graphicx}
\usepackage{listings}
\usepackage[binary,amssymb]{SIunits}
\usepackage[today,nofancy]{svninfo}
\usepackage[natbib,style=alphabetic,maxbibnames=99,backend=bibtex8]{biblatex}
\addbibresource{literature.bib}
\usepackage[listings,varioref,prettyref]{miunmisc}

\svnInfo $Id$

\mode<presentation>
{
  \usetheme{Frankfurt}
  \setbeamercovered{transparent}
  \usecolortheme{seagull}
}
%\def\newblock{\hskip .11em plus .33em minus .07em}
\setbeamertemplate{footline}{\insertframenumber}

\title{%
  Virtual Memory
}
\author{Daniel Bosk\footnote{%
	\tiny
  This work is licensed under the Creative Commons Attribution-ShareAlike 3.0 
  Unported license.
	To view a copy of this license, visit 
	\url{http://creativecommons.org/licenses/by-sa/3.0/}.
}}
\institute[MIUN ICS]{%
  Department of Information and Communication Systems (ICS),\\
  Mid Sweden University, Sundsvall.
}
\date{\svnId}


\AtBeginSection[]{%
  \begin{frame}<beamer>{Overview}
		\tableofcontents[currentsection]
	\end{frame}
}

\begin{document}

\begin{frame}
  \titlepage
\end{frame}

\begin{frame}{Overview}
	\tableofcontents
	% You might wish to add the option [pausesections]
\end{frame}

\begin{frame}{Literature}
  % $Id$
% Author:	Daniel Bosk <daniel.bosk@miun.se>
This assignment covers part three of the book 
\cite{Silberschatz2013intl,Silberschatz2013osc}, i.e.\ memory management.
Hence, before attempting this assignment you should have read chapters 8 and 9, 
``Memory-Management Strategies'' and ``Virtual-Memory Management'', 
respectively.

\end{frame}


% Since this a solution template for a generic talk, very little can
% be said about how it should be structured. However, the talk length
% of between 15min and 45min and the theme suggest that you stick to
% the following rules:  

% - Exactly two or three sections (other than the summary).
% - At *most* three subsections per section.
% - Talk about 30s to 2min per frame. So there should be between about
%   15 and 30 frames, all told.


\section{Virtual Memory}

\subsection{Huge logical but small physical}

\begin{frame}{\insertsubsectionhead}
  \begin{itemize}
    \item What if we want to run a program which is larger than the physical 
      memory?

    \item When we run a program we don't necessarily use all of them, some 
      parts of the code, e.g.\ error handling, isn't always used.

    \item Then why load the entire program into memory?
      Why not load only the pages of a program used by a process?

    \item This is the idea of virtual memory and demand paging.

  \end{itemize}
\end{frame}

\begin{frame}{\insertsubsectionhead}
  \begin{itemize}
    \item We've already separated the logical address space from the physical 
      address space by paging and segmentation.

    \item Now we separate it further by virtual memory.

    \item Virtual memory means we pretend as if we have a huge memory, the 
      virtual memory.

    \item To be able to work with this we need it in physical memory, hence we 
      place different parts of it in physical memory at any given time.

  \end{itemize}
\end{frame}

\begin{frame}{\insertsubsectionhead}
  \begin{figure}
    \includegraphics[height=0.7\textheight]{vmem.eps}
    \caption{An illustration of the idea of a larger virtual memory than 
      physical memory.
      Image: \cite{Silberschatz2013osc}.}
  \end{figure}
\end{frame}

\begin{frame}{\insertsubsectionhead}
  \begin{itemize}
    \item Actually, with this approach \emph{each} process can have a huge 
      virtual memory, seeming to the process that it has all available memory 
      to itself.
  \end{itemize}
\end{frame}

\begin{frame}{\insertsubsectionhead}
  \begin{figure}
    \includegraphics[height=0.7\textheight]{vaddrspace.eps}
    \caption{A process's virtual address space.
      Image: \cite{Silberschatz2013osc}.}
  \end{figure}
\end{frame}

\begin{frame}{\insertsubsectionhead}
  \begin{itemize}
    \item With this huge gap in the middle we can fit a huge stack and an 
      enormous heap.

    \item We can also use the space for other purposes.

    \item We can map in various shared pages there.

  \end{itemize}
\end{frame}

\begin{frame}{\insertsubsectionhead}
  \begin{figure}
    \includegraphics[height=0.7\textheight]{sharedpages.eps}
    \caption{Shared library using virtual memory.
      Image: \cite{Silberschatz2013osc}.}
  \end{figure}
\end{frame}


\section{Demand Paging}

\subsection{What is demand paging?}

\begin{frame}{\insertsubsectionhead}
  \begin{figure}
    \includegraphics[height=0.7\textheight]{swapping.eps}
    \caption{The principle of swapping as seen before.
      Image: \cite{Silberschatz2013osc}.}
  \end{figure}
\end{frame}

\begin{frame}{\insertsubsectionhead}
  \begin{itemize}
    \item What if we swapped in and out only parts of the programs?

    \item This is what a lazy swapper does.

    \item A swapper works with entire processes.

    \item As we now see a process as a sequence of pages, we use a \emph{pager} 
      to swap in and out pages of the process.

    \item For this kind of pager we need some form of hardware support to 
      detect when a page is needed but not in memory, we use the valid--invalid 
      bits from before.

  \end{itemize}
\end{frame}

\subsection{Demand paging}

\begin{frame}{\insertsubsectionhead}
  \begin{figure}
    \includegraphics[height=0.7\textheight]{lazyswap.eps}
    \caption{Page table with some pages in memory and some on disk.
      Image: \cite{Silberschatz2013osc}.}
  \end{figure}
\end{frame}

\begin{frame}{\insertsubsectionhead}
  \begin{itemize}
    \item As long as pages are allocated to frames and are in memory everything 
      works as before.

    \item What happens when the process refers to a page not in memory?

    \item Then the memory unit traps to the OS, it generates a page fault.

  \end{itemize}
\end{frame}

\begin{frame}{\insertsubsectionhead}
  \begin{figure}
    \includegraphics[height=0.7\textheight]{pagefault.eps}
    \caption{Procedure when a page fault occurs.
      Image: \cite{Silberschatz2013osc}.}
  \end{figure}
\end{frame}

\begin{frame}{\insertsubsectionhead}
  \begin{itemize}
    \item Pure demand paging means we never bring a page into memory until it's 
      needed (by generating a page fault).

    \item The hardware requirements for demand paging are the same as for 
      paging (with valid--invalid bits) and swapping.

    \item However, it also needs the ability to restart any instruction in the 
      CPU after a page fault.
  \end{itemize}
\end{frame}

\begin{frame}{\insertsubsectionhead}
  \begin{itemize}
    \item How is the performance of demand paging?

    \item Usually the memory access time is around 10 to 200 nanoseconds.

    \item With no page faults we keep the same performance as usual.

    \item We compute the effective access time as we did before for the TLB.

  \end{itemize}
\end{frame}

\begin{frame}{\insertsubsectionhead}
  \begin{block}{Effective access time}
    The effective access time \(t\) is \[
      t = (1-p)\times t_m + p\times t_p,
    \] where \(p\) is the probability of a page fault, \(t_m\) is the memory 
    access time and \(t_p > 2 t_m\) is the page-fault time.
  \end{block}
\end{frame}

\begin{frame}{\insertsubsectionhead}
  \begin{example}
    We have \(t_m = \unit{200}{\nano\second}\) and \(t_p 
    = \unit{8}{\milli\second} = \unit{8000000}{\nano\second}\).
    We want \(t < \unit{220}{\nano\second}\), i.e.\ less than 10 percent loss.
    \begin{eqnarray}
      t &>& (1-p)\times t_m + p\times t_p \\
      220 &>& (1-p)\times 200 + p\times 8000000 \\
      220 &>& 200 - p\times 200 + p\times 8000000 \\
      220 &>& 200 + p\times (8000000 - 200) \\
      220 &>& 200 + 7999800\times p \\
      20 &>& 7999800\times p \\
      \frac{20}{7999800} = 0.0000025 &>& p
    \end{eqnarray}
  \end{example}
\end{frame}

\subsection{Copy-on-Write}

\begin{frame}{\insertsubsectionhead}
  \begin{figure}
    \includegraphics[width=\textwidth]{copyonwrite.eps}
    \caption{Processes sharing pages.
      Image: \cite{Silberschatz2013osc}.}
  \end{figure}
\end{frame}

\begin{frame}{\insertsubsectionhead}
  \begin{figure}
    \includegraphics[width=\textwidth]{copyonwritechange.eps}
    \caption{One of the processes changes one of the pages.
      Image: \cite{Silberschatz2013osc}.}
  \end{figure}
\end{frame}


\section{Page Replacement}

\subsection{Need for page replacement}

\begin{frame}{\insertsubsectionhead}
  \begin{figure}
    \includegraphics[height=0.7\textheight]{needreplace.eps}
    \caption{Need for page replacement.
      Image: \cite{Silberschatz2013osc}.}
  \end{figure}
\end{frame}

\begin{frame}{\insertsubsectionhead}
  \begin{figure}
    \includegraphics[height=0.7\textheight]{pagereplace.eps}
    \caption{A page replacement happening.
      Image: \cite{Silberschatz2013osc}.}
  \end{figure}
\end{frame}

\begin{frame}{\insertsubsectionhead}
  \begin{itemize}
    \item Need to select a victim frame.
    \item Different page-replacement algorithms for this.
    \item Some pages are more expensive to replace.
    \item If they're not changed we can use the copy on disk already.
    \item If they're changed we need to write the new version to disk again.
  \end{itemize}
\end{frame}

\subsection{Page-replacement algorithms}

\begin{frame}{\insertsubsectionhead}
  \begin{itemize}
    \item We can use a page-reference string to test an algorithm through 
      a simulator.

    \item For (some of) these different algorithms there is a simulator, this 
      one is included with the lab.

    \item You have a C version and a Python version.

    \item cpager: \term{cd cpager \&\& make pager \&\& ./pager}

    \item pypager: \term{cd pypager \&\& ./pypager}

  \end{itemize}
\end{frame}

\subsection{FIFO page replacement}

\begin{frame}{\insertsubsectionhead}
\end{frame}

\subsection{Optimal page replacement}

\begin{frame}{\insertsubsectionhead}
\end{frame}

\subsection{LRU page replacement}

\begin{frame}{\insertsubsectionhead}
\end{frame}

\subsection{LRU-approximation page replacement}

\begin{frame}{\insertsubsectionhead}{Additional reference bits}
\end{frame}

\begin{frame}{\insertsubsectionhead}{Second-Change algorithm}
\end{frame}

\begin{frame}{\insertsubsectionhead}{Enhanced Second-Change algorithm}
\end{frame}


\section{Frame Allocation}

\subsection{Allocation between processes}

\begin{frame}{\insertsubsectionhead}
  \begin{itemize}
    \item How many frames should each process be given?
  \end{itemize}
\end{frame}

\begin{frame}{\insertsubsectionhead}
\end{frame}


%%%%%%%%%%%%%%%%%%%%%%

\begin{frame}[allowframebreaks]{Referenser}
  \small
  \printbibliography
\end{frame}

\end{document}

