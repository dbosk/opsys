% $Id$
% Author:	Daniel Bosk <daniel.bosk@miun.se>
%\documentclass[handout]{beamer}
\documentclass{beamer}
\usepackage[utf8]{inputenc}
\usepackage[T1]{fontenc}
\usepackage[english,swedish]{babel}
\usepackage{url}
\usepackage{varioref,prettyref}
\usepackage{graphicx}
\usepackage[today,nofancy]{svninfo}
\usepackage[natbib,style=alphabetic,maxbibnames=99,backend=bibtex8]{biblatex}
\addbibresource{literature.bib}
\usepackage[varioref,prettyref]{miunmisc}

\svnInfo $Id$

\mode<presentation>
{
  \usetheme{Frankfurt}
  \setbeamercovered{transparent}
  \usecolortheme{seagull}
}
%\def\newblock{\hskip .11em plus .33em minus .07em}
\setbeamertemplate{footline}{\insertframenumber}

\title{%
  Physical Memory
}
\author{Daniel Bosk\footnote{%
	\tiny
  This work is licensed under the Creative Commons Attribution-ShareAlike 3.0 
  Unported license.
	To view a copy of this license, visit 
	\url{http://creativecommons.org/licenses/by-sa/3.0/}.
}}
\institute[MIUN ICS]{%
  Department of Information and Communication Systems (ICS),\\
  Mid Sweden University, Sundsvall.
}
\date{\svnId}


\AtBeginSection[]{%
  \begin{frame}<beamer>{Overview}
		\tableofcontents[currentsection]
	\end{frame}
}

\begin{document}

\begin{frame}
  \titlepage
\end{frame}

\begin{frame}{Overview}
	\tableofcontents
	% You might wish to add the option [pausesections]
\end{frame}

\begin{frame}{Literature}
  % $Id$
% Author:	Daniel Bosk <daniel.bosk@miun.se>
This assignment covers part three of the book 
\cite{Silberschatz2013intl,Silberschatz2013osc}, i.e.\ memory management.
Hence, before attempting this assignment you should have read chapters 8 and 9, 
``Memory-Management Strategies'' and ``Virtual-Memory Management'', 
respectively.

\end{frame}


% Since this a solution template for a generic talk, very little can
% be said about how it should be structured. However, the talk length
% of between 15min and 45min and the theme suggest that you stick to
% the following rules:  

% - Exactly two or three sections (other than the summary).
% - At *most* three subsections per section.
% - Talk about 30s to 2min per frame. So there should be between about
%   15 and 30 frames, all told.


\section{Hardware}

\subsection{Basic hardware}

\begin{frame}{\insertsubsectionhead}
  \begin{figure}
    \includegraphics[height=0.7\textheight]{membaselimit.pdf}
    \caption{Memory layout with base and limit for logical address space.
      Image: \cite{Silberschatz2013osc}.}
  \end{figure}
\end{frame}

\begin{frame}{\insertsubsectionhead}
  \begin{figure}
    \includegraphics[width=\textwidth]{hwbaselimit.pdf}
    \caption{A schematic of hardware protection for base and limit memory 
      protection.
      Image: \cite{Silberschatz2013osc}.}
  \end{figure}
\end{frame}


\section{Address Binding}

\subsection{Binding memory in a process}

\begin{frame}{\insertsubsectionhead}
  \begin{figure}
    \includegraphics[height=0.7\textheight]{loadlink.pdf}
    \caption{An overview of all steps to create a program.
      Image: \cite{Silberschatz2013osc}.}
  \end{figure}
\end{frame}

\subsection{Relocation registers}

\begin{frame}{\insertsubsectionhead}
  \begin{figure}
    \includegraphics[height=0.7\textheight]{relocation.pdf}
    \caption{An illustration of a relocation register.
      Image: \cite{Silberschatz2013osc}.}
  \end{figure}
\end{frame}

\begin{frame}{\insertsubsectionhead}
  \begin{figure}
    \includegraphics[width=\textwidth]{safereloc.pdf}
    \caption{A relocation register with base and limit properties.
      Image: \cite{Silberschatz2013osc}.}
  \end{figure}
\end{frame}


\section{Memory Allocation}

\subsection{Contiguous Allocation}

\begin{frame}{\insertsubsectionhead}
  \begin{itemize}
    \item First-fit
    \item Best-fit
    \item Worst-fit
  \end{itemize}
\end{frame}

\begin{frame}{\insertsubsectionhead}
  \begin{itemize}
    \item 50-percent rule
  \end{itemize}
\end{frame}

\subsection{Non-Contiguous Allocation}

\begin{frame}{\insertsubsectionhead}
  \begin{itemize}
    \item Paging
    \item Segmentation
  \end{itemize}
\end{frame}

\subsection{Paging}

\begin{frame}{\insertsubsectionhead}
  \begin{itemize}
    \item Split the logical address into two parts:
      \begin{itemize}
        \item page number, and
        \item page offset (displacement).
      \end{itemize}

    \item Fix the size of the pages in bits of the address, i.e.\ the size is 
      a power of two.
      E.g.\ page size is the last 10 bits, then each page will be \(2^{10}\) 
      bytes or 1 KB in size.

    \item The size of the page must be equal to the size of the physical 
      counterpart, the frame.
      Hence the page size is determined by the hardware.

    \item Each page is mapped to a frame, this mapping is kept in a page table.

  \end{itemize}
\end{frame}

\begin{frame}{\insertsubsectionhead}
  \begin{figure}
    \includegraphics[height=0.7\textheight]{pagemap.pdf}
    \caption{An overview of paging.
      Image: \cite{Silberschatz2013osc}.}
  \end{figure}
\end{frame}

\begin{frame}{\insertsubsectionhead}
  \begin{figure}
    \includegraphics[height=0.7\textheight]{pagetable.pdf}
    \caption{A pagetable.
      Image: \cite{Silberschatz2013osc}.}
  \end{figure}
\end{frame}

\begin{frame}{\insertsubsectionhead}
  \begin{figure}
    \includegraphics[height=0.7\textheight]{pagealloc.pdf}
    \caption{Allocated memory using paging.
      Image: \cite{Silberschatz2013osc}.}
  \end{figure}
\end{frame}

\begin{frame}{\insertsubsectionhead}
  \begin{figure}
    \includegraphics[height=0.7\textheight]{pagenew.pdf}
    \caption{Before and after allocating a new process to memory.
      Image: \cite{Silberschatz2013osc}.}
  \end{figure}
\end{frame}

\begin{frame}{\insertsubsectionhead}
  \begin{itemize}
    \item The page table is usually stored by the OS in the process's PCB.

    \item The dispatcher loads the process's page table into the hardware, if 
      the page table is sufficiently small.
      
    \item Bigger tables must be kept in the memory, then a page table base 
      register points to the page table's position in memory.

    \item However, this requires a memory lookup, i.e.\ each memory reference 
      requires two references!

    \item The solution is to add a translation look-aside buffer (TLB), a very 
      fast hardware cache.

  \end{itemize}
\end{frame}

\begin{frame}{\insertsubsectionhead}
  \begin{figure}
    \includegraphics[height=0.7\textheight]{tlb.pdf}
    \caption{Paging with a TLB.
      Image: \cite{Silberschatz2013osc}.}
  \end{figure}
\end{frame}

\begin{frame}{\insertsubsectionhead}
  \begin{itemize}
    \item Some TLBs has address-space identifications (ASIDs).

    \item ASIDs allows the TLB to handle pages for different processes 
      simultaneously.

    \item Hence we don't need to clear it every context switch.

  \end{itemize}
\end{frame}

\begin{frame}{\insertsubsectionhead}
  \begin{itemize}
    \item One can compute the effective access time for memory access.

    \item Effective access time is the average time it takes for each memory 
      access, i.e.\ with regards to TLB misses and hits.

  \end{itemize}
\end{frame}

\begin{frame}{\insertsubsectionhead}
  \begin{block}{Effective access time}
    We can compute the effective access time \(t\) by \[
      t = (1-p)\times (t_t + t_m) + p\times (t_t + t_m + t_m),
    \] where \(p\) is the probability of a TLB miss, \(t_t\) is the time for 
    a TLB lookup, \(t_m\) is the time for a memory access.
  \end{block}
\end{frame}

\begin{frame}{\insertsubsectionhead}
  \begin{itemize}
    \item We must have some protection for the pages of a process.

    \item What happens when the process's logical address space is smaller than 
      the possible physical address space.
  \end{itemize}
\end{frame}

\begin{frame}{\insertsubsectionhead}
  \begin{figure}
    \includegraphics[height=0.7\textheight]{pagevalid.pdf}
    \caption{Paging with valid-bits.
      Image: \cite{Silberschatz2013osc}.}
  \end{figure}
\end{frame}

\begin{frame}{\insertsubsectionhead}
  \begin{itemize}
    \item We can also add additional bits like this.

    \item We can have bits for specifying whether reading, writing or execution 
      is allowed on these pages.

    \item This way we can implement execution prevention for certain memory 
      regions, e.g.\ where the web-browser stores data downloaded from the Web.

    \item This way several processes can share read-only pages, hence requiring 
      less memory.

  \end{itemize}
\end{frame}

\begin{frame}{\insertsubsectionhead}
  \begin{figure}
    \includegraphics[height=0.7\textheight]{sharedpage.pdf}
    \caption{An example of shared pages.
      Image: \cite{Silberschatz2013osc}.}
  \end{figure}
\end{frame}

\begin{frame}{\insertsubsectionhead}
  \begin{itemize}
    \item What happens when we get huge logical address spaces, e.g.\ 32 or 
      64-bit addresses?

    \item With a page size of 4 KB (i.e.\ 12 bits) the worst case scenario is 
      a page table of 4 MB in size -- for each process!

    \item If we need to allocate this contiguously in memory we're back to the 
      original problem we wanted to solve.

    \item Well, we have a solution for that: paging!

    \item We page the page table ...

  \end{itemize}
\end{frame}

\begin{frame}{\insertsubsectionhead}
  \begin{figure}
    \includegraphics[width=\textwidth]{pagehierbig.pdf}
    \caption{Hierarchical paging.
      Image: \cite{Silberschatz2013osc}.}
  \end{figure}
\end{frame}

\begin{frame}{\insertsubsectionhead}
  \begin{figure}
    \includegraphics[height=0.7\textheight]{pagehier.pdf}
    \caption{Hierarchical paging, overview.
      Image: \cite{Silberschatz2013osc}.}
  \end{figure}
\end{frame}

\begin{frame}{\insertsubsectionhead}
  \begin{itemize}
    \item Now each part of the page table fits in a page.
    \item Hence we don't need contiguous allocation anymore.
    \item However, this doesn't work for spaces even as small as 64 bits.
  \end{itemize}
\end{frame}

\begin{frame}{\insertsubsectionhead}
  \begin{figure}
    \includegraphics[height=0.7\textheight]{pagehash.pdf}
    \caption{Paging using hashing.
      Image: \cite{Silberschatz2013osc}.}
  \end{figure}
\end{frame}

\begin{frame}{\insertsubsectionhead}
  \begin{itemize}
    \item Another way of saving space is to use an inverted page table.
    \item Here we have one entry per frame, not per page.
    \item Then we map which process has allocated it and to which page.
  \end{itemize}
\end{frame}

\begin{frame}{\insertsubsectionhead}
  \begin{figure}
    \includegraphics[height=0.7\textheight]{pageidx.pdf}
    \caption{Inverted paging.
      Image: \cite{Silberschatz2013osc}.}
  \end{figure}
\end{frame}

\subsection{Segmentation}

\begin{frame}{\insertsubsectionhead}
  \begin{figure}
    \includegraphics[height=0.7\textheight]{segments.pdf}
    \caption{Segments.
      Image: \cite{Silberschatz2013osc}.}
  \end{figure}
\end{frame}

\begin{frame}{\insertsubsectionhead}
  \begin{figure}
    \includegraphics[height=0.7\textheight]{seghw.pdf}
    \caption{Segmentation hardware.
      Image: \cite{Silberschatz2013osc}.}
  \end{figure}
\end{frame}

\begin{frame}{\insertsubsectionhead}
  \begin{figure}
    \includegraphics[height=0.7\textheight]{segmap.pdf}
    \caption{Segment mapping.
      Image: \cite{Silberschatz2013osc}.}
  \end{figure}
\end{frame}

%\begin{frame}{\insertsubsectionhead}
%  \begin{figure}
%    \includegraphics[width=\textwidth]{pageseg.pdf}
%    \caption{Segmentation and paging.
%      Image: \cite{Silberschatz2013osc}.}
%  \end{figure}
%\end{frame}


%%%%%%%%%%%%%%%%%%%%%%

\begin{frame}[allowframebreaks]{Referenser}
  \small
  \printbibliography
\end{frame}

\end{document}

