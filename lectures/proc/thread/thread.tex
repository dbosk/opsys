% $Id$
% Author:	Daniel Bosk <daniel.bosk@miun.se>
%\documentclass[handout]{beamer}
\documentclass{beamer}
\usepackage[utf8]{inputenc}
\usepackage[T1]{fontenc}
\usepackage[swedish,english]{babel}
\usepackage{url}
\usepackage{graphicx}
\usepackage{color}
\usepackage{listings}
\usepackage[today,nofancy]{svninfo}
\usepackage[natbib,style=alphabetic,maxbibnames=99,backend=bibtex8]{biblatex}
\addbibresource{literature.bib}
\usepackage[listings]{miunmisc}

\svnInfo $Id$

\mode<presentation>
{
  \usetheme{Frankfurt}
  \setbeamercovered{transparent}
  \usecolortheme{seagull}
}
%\def\newblock{\hskip .11em plus .33em minus .07em}
\setbeamertemplate{footline}{\insertframenumber}

\title{%
  Threads
}
\author{Daniel Bosk\footnote{%
	\tiny
  This work is licensed under the Creative Commons Attribution-ShareAlike 3.0 
  Unported license.
	To view a copy of this license, visit 
	\url{http://creativecommons.org/licenses/by-sa/3.0/}.
}}
\institute[MIUN ICS]{%
  Department of Information and Communication Systems (ICS),\\
  Mid Sweden University, Sundsvall.
}
\date{\svnId}


\AtBeginSection[]{%
  \begin{frame}<beamer>{Overview}
		\tableofcontents[currentsection]
	\end{frame}
}

\begin{document}

\begin{frame}
  \titlepage
\end{frame}

\begin{frame}{Overview}
	\tableofcontents
	% You might wish to add the option [pausesections]
\end{frame}

\begin{frame}{Literature}
  % $Id$
This lecture gives an overview of the technicalities of the file system.
This is covered by Chapter 10 ``File System'' in \cite{Silberschatz2009osc}, or 
Chapter 11 ``File-System Interface'' in 
\cite{Silberschatz2013osc,Silberschatz2013intl}.

\end{frame}


% Since this a solution template for a generic talk, very little can
% be said about how it should be structured. However, the talk length
% of between 15min and 45min and the theme suggest that you stick to
% the following rules:  

% - Exactly two or three sections (other than the summary).
% - At *most* three subsections per section.
% - Talk about 30s to 2min per frame. So there should be between about
%   15 and 30 frames, all told.


\section{Threads}

\subsection{What is a thread?}

\begin{frame}{\insertsubsectionhead}
  \begin{itemize}
    \item The process is what we consider the smallest entity of execution.
    \item This contains the program code, the variables (data), etc.
    \item A process has traditionally only one \emph{thread} of execution, 
      called a heavy-weight process.
    \item Now, however, we'll extend this with more threads of execution.
  \end{itemize}
\end{frame}

\begin{frame}{\insertsubsectionhead}
  \begin{figure}
    \includegraphics[height=0.7\textheight]{threads.pdf}
    \caption{Single-threaded vs.\ multi-threaded process.
      Image: \cite{Silberschatz2013osc}.}
  \end{figure}
\end{frame}

\begin{frame}{\insertsubsectionhead}
  \begin{itemize}
    \item Multiple threads of execution means we can do many things 
      simultaneously.

    \item Have to be careful though, you never know who is changing something 
      in the program at a given time.

  \end{itemize}
\end{frame}

\subsection{Benefits}

\begin{frame}{\insertsubsectionhead}
  \begin{itemize}
    \item Responsiveness
    \item Resource sharing
    \item Economy (context switching, process creation)
    \item Scalabiblity
  \end{itemize}
\end{frame}

\begin{frame}{\insertsubsectionhead}{Scalability}
  \begin{figure}
    \includegraphics[width=0.45\textwidth]{single-core.pdf}
    \hfill
    \includegraphics[width=0.45\textwidth]{multi-core.pdf}
    \caption{A single-core and multi-core execution of two threads.
      Image: \cite{Silberschatz2013osc}.}
  \end{figure}
\end{frame}

\subsection{Issues}

\begin{frame}{\insertsubsectionhead}
  \begin{itemize}
    \item What happens with fork(2) and exec(2)?
      The calling thread only, or all threads?

    \item exec(2) usually replaces all threads.

    \item fork(2) can be different; in some cases only the calling thread is 
      reasonable to fork, in other cases all of them.

    \item Programming becomes more complex, introducing many potential bugs.

  \end{itemize}
\end{frame}

\section[Implementation]{Implementation of threads}

\subsection{Types of threads}

\begin{frame}{\insertsubsectionhead}
  \begin{description}
    \item[User thread] This is what we imagine from the image above.
      This executes in user-mode, and is invisible to the kernel.

    \item[Kernel thread] These are threads part of the kernel.
      These does not necessarily execute all code in kernel-mode though.
  \end{description}
\end{frame}

\begin{frame}{\insertsubsectionhead}
  \begin{itemize}
    \item User-threads are mapped to kernel-threads.
    \item This mapping can be done in several ways:
      \begin{itemize}
        \item One-to-one,
        \item Many-to-one,
        \item Many-to-many.
      \end{itemize}
  \end{itemize}
\end{frame}

\subsection{The different models}

\begin{frame}{\insertsubsectionhead}
  \begin{figure}
    \includegraphics[width=\textwidth]{one-to-one.pdf}
    \caption{The one-to-one model.
      Image: \cite{Silberschatz2013osc}.}
  \end{figure}
\end{frame}

\begin{frame}{\insertsubsectionhead}
  \begin{itemize}
    \item Good in that it's the OS handling all scheduling.
    \item Bad in that it requires as many kernel-threads as user-threads.
  \end{itemize}
\end{frame}

\begin{frame}{\insertsubsectionhead}
  \begin{figure}
    \includegraphics[height=0.7\textheight]{many-to-one.pdf}
    \caption{The many-to-one model.
      Image: \cite{Silberschatz2013osc}.}
  \end{figure}
\end{frame}

\begin{frame}{\insertsubsectionhead}
  \begin{itemize}
    \item Bad in that if one thread blocks, the whole process blocks, and thus 
      all threads block.

    \item Can be combined with the one-to-one model though, then threads can be 
      partitioned.

  \end{itemize}
\end{frame}

\begin{frame}{\insertsubsectionhead}
  \begin{figure}
    \includegraphics[height=0.7\textheight]{many-to-many.pdf}
    \caption{The many-to-many model.
      Image: \cite{Silberschatz2013osc}.}
  \end{figure}
\end{frame}

\begin{frame}{\insertsubsectionhead}
  \begin{itemize}
    \item The many-to-many model requires a light-weight process (LWP).

    \item To the process this appears as a processor on which it can schedule 
      its threads.

    \item I.e.\ the thread library does the scheduling of its threads, as is 
      the case in many-to-one also.

    \item The kernel schedules all threads in the one-to-one model.
  \end{itemize}
\end{frame}

\begin{frame}{\insertsubsectionhead}
  \begin{figure}
    \includegraphics[height=0.7\textheight]{lwp.pdf}
    \caption{Structure including an LWP.
      Image: \cite{Silberschatz2013osc}.}
  \end{figure}
\end{frame}

\begin{frame}{\insertsubsectionhead}
  \begin{itemize}
    \item The many-to-many model has the advantage that it has none of the 
      problems of the other models.

    \item It can have an arbitrarily large amount of threads.

    \item If one thread blocks, i.e.\ one LWP is blocked, just schedule another 
      LWP.
  \end{itemize}
\end{frame}

\begin{frame}{\insertsubsectionhead}
  \begin{figure}
    \includegraphics[height=0.7\textheight]{two-level.pdf}
    \caption{The two-level model.
      Image: \cite{Silberschatz2013osc}.}
  \end{figure}
\end{frame}

\section{Signals}

\subsection{What are signals?}

\begin{frame}{\insertsubsectionhead}
  \begin{itemize}
    \item A signal is lika an interrupt for a process.

    \item The OS sends the process a signal.

    \item The process must interrupt its current work to handle the signal, 
      then it may return to previous work.

    \item Examples include ``Division by zero'', ``Illegal memory access'' and 
      hitting Ctrl+C keyboard combination.

  \end{itemize}
\end{frame}

\begin{frame}{\insertsubsectionhead}
  \begin{enumerate}
    \item Program executes division by zero.
    \item CPU generates interrupt, calls OS interrupt handler.
    \item OS interrupt handler notes the process currently using the CPU.
    \item OS generates as signal to the process.
    \item Process's signal handler executes.
  \end{enumerate}
\end{frame}

\begin{frame}{\insertsubsectionhead}
  \begin{itemize}
    \item With a multithreaded process, which thread should handle a signal 
      from the OS?

    \item This can usually be specified by the programmer.

  \end{itemize}
\end{frame}


\section{Playing with Threads in Python}

\subsection{With race-condition}

\begin{frame}[fragile,allowframebreaks]{\insertsubsectionhead}
  \lstinputlisting[style=code,language=Python]{race.py}
\end{frame}

\subsection{Without race-condition}

\begin{frame}[fragile,allowframebreaks]{\insertsubsectionhead}
  \lstinputlisting[style=code,language=Python]{norace.py}
\end{frame}



%%%%%%%%%%%%%%%%%%%%%%

\begin{frame}[allowframebreaks]{Referenser}
  \small
  \printbibliography
\end{frame}

\end{document}

