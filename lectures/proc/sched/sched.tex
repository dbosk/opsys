% $Id$
% Author:	Daniel Bosk <daniel.bosk@miun.se>
%\documentclass[handout]{beamer}
\documentclass{beamer}
\usepackage[utf8]{inputenc}
\usepackage[T1]{fontenc}
\usepackage[swedish,english]{babel}
\usepackage{url}
\usepackage{varioref,prettyref}
\usepackage{graphicx}
\usepackage[binary,squaren]{SIunits}
\usepackage{booktabs}
\usepackage{listings}
\usepackage[today,nofancy]{svninfo}
\usepackage[natbib,style=alphabetic,maxbibnames=99,backend=bibtex8]{biblatex}
\addbibresource{literature.bib}
\usepackage[listings,varioref,prettyref]{miunmisc}

\svnInfo $Id$

\mode<presentation>
{
  \usetheme{Frankfurt}
  \setbeamercovered{transparent}
  \usecolortheme{seagull}
}
%\def\newblock{\hskip .11em plus .33em minus .07em}
\setbeamertemplate{footline}{\insertframenumber}

\title{%
  Scheduling
}
\author{Daniel Bosk\footnote{%
	\tiny
  This work is licensed under the Creative Commons Attribution-ShareAlike 3.0 
  Unported license.
	To view a copy of this license, visit 
	\url{http://creativecommons.org/licenses/by-sa/3.0/}.
}}
\institute[MIUN ICS]{%
  Department of Information and Communication Systems (ICS),\\
  Mid Sweden University, Sundsvall.
}
\date{\svnId}


\AtBeginSection[]{%
  \begin{frame}<beamer>{Overview}
		\tableofcontents[currentsection]
	\end{frame}
}

\begin{document}

\begin{frame}
  \titlepage
\end{frame}

\begin{frame}{Overview}
	\tableofcontents
	% You might wish to add the option [pausesections]
\end{frame}

\begin{frame}{Literature}
  % $Id$
This lecture gives an overview of the technicalities of the file system.
This is covered by Chapter 10 ``File System'' in \cite{Silberschatz2009osc}, or 
Chapter 11 ``File-System Interface'' in 
\cite{Silberschatz2013osc,Silberschatz2013intl}.

\end{frame}


% Since this a solution template for a generic talk, very little can
% be said about how it should be structured. However, the talk length
% of between 15min and 45min and the theme suggest that you stick to
% the following rules:  

% - Exactly two or three sections (other than the summary).
% - At *most* three subsections per section.
% - Talk about 30s to 2min per frame. So there should be between about
%   15 and 30 frames, all told.

\section[Scheduling]{Process Scheduling}

\subsection{CPU Scheduling}

\begin{frame}{\insertsubsectionhead}
  \begin{itemize}
    \item CPU-scheduling is at the heart of multiprogramming.

    \item It concerns the short-term scheduler.

    \item It is used to schedule which process is allowed to execute its 
      instructions on the CPU at any given time.

    \item If the operating system supports kernel threads it's kernel threads, 
      not processes, being scheduled.

  \end{itemize}
\end{frame}

\begin{frame}{\insertsubsectionhead}
  \begin{itemize}
    \item The short-term scheduler selects which process to allocate the CPU to 
      next.

    \item The dispatcher is the function which performs
      the context switch,
      switching to user mode,
      and restarting the process on the correct instruction.

    \item It's important that these functions execute quickly, as they might be 
      executed every 10 to 100 milliseconds.

  \end{itemize}
\end{frame}

\subsection{CPU-I/O Burst Cycle}

\begin{frame}{\insertsubsectionhead}
  \begin{itemize}
    \item Process execution can be divided into two classes.
      \begin{enumerate}
        \item CPU-bursts:
          The process executes on the CPU, actually executing instructions, 
          e.g.\ performing computations.

        \item I/O-bursts:
          The process spends its time reading from or writing to various 
          devices, i.e.\ the process spends a lot of time waiting for the 
          I/O-devices.

      \end{enumerate}
  \end{itemize}
\end{frame}

\begin{frame}{\insertsubsectionhead}
  \begin{figure}
    \includegraphics[height=0.7\textheight]{burstcycle.pdf}
    \caption{An illustration of the CPU-I/O burst cycle for a process.
      Image: \cite{Silberschatz2013osc}.}
  \end{figure}
\end{frame}

\begin{frame}{\insertsubsectionhead}
  \begin{figure}
    \includegraphics[height=0.7\textheight]{burstdiag.pdf}
    \caption{A diagram of a typical ratio between CPU and I/O bursts.
      Image: \cite{Silberschatz2013osc}.}
  \end{figure}
\end{frame}

\subsection{Preemptive and Non-Preemptive Scheduling}

\begin{frame}{\insertsubsectionhead}
  \begin{itemize}
    \item Scheduling can be divided into the two classes preemptive and 
      non-preemptive scheduling.

    \item Non-Preemptive or cooperative scheduling leaves it up to the 
      processes to give up the processor, e.g.\ through blocking on I/O.

    \item Preemptive does abort one process in favour of another process, e.g.\ 
      every process may have 100 milliseconds each in turn.

  \end{itemize}
\end{frame}

\begin{frame}{\insertsubsectionhead}
  \begin{itemize}
    \item Preemptive is most prominent.
    \item Requires special hardware though, e.g.\ timers.
    \item Synchronisation issuse also arises, e.g.\ race conditions.
  \end{itemize}
\end{frame}


\section{Algorithms}

\subsection{Scheduling Criteria}

\begin{frame}{\insertsubsectionhead}
  \begin{description}
    \item[CPU utilisation] We can to keep the CPU as busy as we possibly can.

    \item[Throughput] One measure of performance is how many processes complete 
      per time unit.

    \item[Turnaround time] Yet another measure is how long it takes from jobs 
      submission to job completion.

    \item[Waiting time] The scheduling algorithm doesn't affect the time it 
      takes for a process to execute instructions or do I/O, it only affects 
      the waiting time.

    \item[Response time] In an interactive system the turnaround time or 
      throughput might not be relevant measures.
      Response time is more suitable as a process might be able to produce some 
      output even before it's done.

  \end{description}
\end{frame}

\subsection{First-Come, First-Served (FCFS)}

\begin{frame}{\insertsubsectionhead}
  \begin{itemize}
    \item A very basic schduling algorithm allocating the CPU to the processes 
      in FIFO order.

    \item The downside of this algorithm is that average waiting time is 
      usually quite long.

  \end{itemize}
\end{frame}

\begin{frame}{\insertsubsectionhead}
  \begin{table}
    \begin{tabular}{cc}
      \textbf{Proc} & \textbf{Burst-time} \\
      \toprule
      \(P_0\)   & 24 \\
      \(P_1\)   & 2 \\
      \(P_2\)   & 4 \\
      \bottomrule
    \end{tabular}
    \caption{An example of processes and their bust-time.}
  \end{table}
\end{frame}

\subsection{Shortest Job First (SJF)}

\begin{frame}{\insertsubsectionhead}
  \begin{itemize}
    \item This algorithm uses the length of the processes' CPU-bursts to 
      determine the order of execution.

    \item The problem with this algorithm is that it's nearly impossible to 
      implement, estimating the length of a CPU-burst is very hard to do 
      accurately.

    \item However, it's provably optimal, giving the minimum average waiting 
      time.

    \item One can do it by estimation, using the lengths of previous bursts to 
      predict the length of the coming one.

  \end{itemize}
\end{frame}

\begin{frame}{\insertsubsectionhead}
  \begin{table}
    \begin{tabular}{cc}
      \textbf{Proc} & \textbf{Burst-time} \\
      \toprule
      \(P_0\)   & 6 \\
      \(P_1\)   & 8 \\
      \(P_2\)   & 7 \\
      \(P_3\)   & 3 \\
      \bottomrule
    \end{tabular}
    \caption{An example of processes and their burst-time.}
  \end{table}
\end{frame}

\subsection{Priority Scheduling}

\begin{frame}{\insertsubsectionhead}
  \begin{itemize}
    \item Priority scheduling is the general form of SJF, if using as priority 
      for a process the inverse predicted length of CPU-burst.

    \item Higher-priority processes are scheduled before lower-priority 
      processes.

    \item Processes having the same priority is scheduled using FCFS.

    \item Major problem with all priority-based algorithms is indefinite 
      blocking, or starvation.

    \item One solution to this is some kind of aging, older processes increase 
      in priority.

  \end{itemize}
\end{frame}

\begin{frame}{\insertsubsectionhead}
  \begin{table}
    \begin{tabular}{ccc}
      \textbf{Proc} & \textbf{Burst-time} & \textbf{Priority} \\
      \toprule
      \(P_0\)   & 10  & 3 \\
      \(P_1\)   & 1   & 1 \\
      \(P_2\)   & 2   & 4 \\
      \bottomrule
    \end{tabular}
    \caption{An example of processes and their burst-time and priorities.}
  \end{table}
\end{frame}

\subsection{Round Robin (RR)}

\begin{frame}{\insertsubsectionhead}
  \begin{itemize}
    \item Uses time-quantum, each process is allocated this much time on the 
      CPU, then it's preempted.

    \item Average waiting-time is often long.

    \item With a time-quantum \(q\) and \(n\) processes, each process executes 
      \(\frac{1}{n} q\) and waits at most \( (n-1)\times q\) time units.

    \item The performance depends on the size of the time-quantum.

  \end{itemize}
\end{frame}

\begin{frame}{\insertsubsectionhead}
  \begin{figure}
    \includegraphics[width=\textwidth]{rr-quantum.pdf}
    \caption{An illustration that smaller time-quantum increases the number of 
      context swtiches.
      Image: \cite{Silberschatz2013osc}.}
  \end{figure}
\end{frame}

\begin{frame}{\insertsubsectionhead}
  \begin{table}
    \begin{tabular}{cc}
      \textbf{Proc} & \textbf{Burst-time} \\
      \toprule
      \(P_0\)   & 24 \\
      \(P_1\)   & 2 \\
      \(P_2\)   & 4 \\
      \bottomrule
    \end{tabular}
    \caption{An example of processes and their bust-time.}
  \end{table}
\end{frame}

\begin{frame}{\insertsubsectionhead}
\end{frame}

\subsection{Multilevel Queue}

\begin{frame}{\insertsubsectionhead}
  \begin{itemize}
    \item We have several ready queues.
    \item Each queue has its own scheduling algorithms.
    \item On top of that we have scheduling among queues.
    \item Processes in lower-priority queues are preempted when a process 
      enters in a higher-priority queue.
  \end{itemize}
\end{frame}

\begin{frame}{\insertsubsectionhead}
  \begin{figure}
    \includegraphics[height=0.7\textheight]{multilevel.pdf}
    \caption{An example of a multilevel queue structure.
      Image: \cite{Silberschatz2013osc}.}
  \end{figure}
\end{frame}

\subsection{Multilevel Feedback Queue}

\begin{frame}{\insertsubsectionhead}
  \begin{itemize}
    \item An extension of previous algorithm in that a process may move between 
      queues depending on its CPU-burst characteristic.

    \item E.g.\ a process using too much CPU-time will be lowered, a process 
      just doing I/O may be higher.

    \item When the I/O-bound process becomes a CPU-bound process it will be 
      lowered again.

  \end{itemize}
\end{frame}

\begin{frame}{\insertsubsectionhead}
  \begin{figure}
    \includegraphics[height=0.7\textheight]{multilvl-feedback.pdf}
    \caption{An example of a multilevel feedback queue system.
      Image: \cite{Silberschatz2013osc}.}
  \end{figure}
\end{frame}


\section[Multiprocessors]{Multiprocessor Scheduling}

\subsection{Asymmetric Multiprocessor Scheduling}

\begin{frame}{\insertsubsectionhead}
  \begin{itemize}
    \item In this kind of setup we have one master server delegating work to 
      the other processors.

    \item I.e.\ one processor (the master) does I/O, scheduling, etc.\ whereas 
      all others execute only user code.
  \end{itemize}
\end{frame}

\subsection{Symmetric Multiprocessor Scheduling}

\begin{frame}{\insertsubsectionhead}
  \begin{itemize}
    \item This is the more common approach.
    \item All processors are peers, executing everything.
    \item I.e.\ all processors does I/O etc.
    \item Every processor schedules its own execution; has perhaps an own ready 
      queue, perhaps a joint one for all processors.
  \end{itemize}
\end{frame}

\begin{frame}{\insertsubsectionhead}
  \begin{itemize}
    \item One problem which arises with SMP is non-uniform memory access 
      (NUMA), i.e.\ when different parts of memory is accessed with different 
      speeds.

    \item Another problem is if a process runs on one processor and is the 
      scheduled on another processor, then the cache is invalid on the previous 
      processor and it must be filled on the new one.

    \item Hence it's bad to move processes between processors.

    \item This is called processor affinity.
  \end{itemize}
\end{frame}

\begin{frame}{\insertsubsectionhead}
  \begin{figure}
    \includegraphics[height=0.7\textheight]{numa.pdf}
    \caption{An illustration of the problem with NUMA.
      Image: \cite{Silberschatz2013osc}.}
  \end{figure}
\end{frame}

\begin{frame}{\insertsubsectionhead}
  \begin{itemize}
    \item We must have some type of load balancing, otherwise all processes 
      might end up on one processor with all other processors idle.

    \item We therefore have soft and hard affinity, to allow moves in some 
      cases but not in others.

    \item Load balancing in SMP can be done with either push or poll migration.

    \item However, they are not mutually exclusive.

  \end{itemize}
\end{frame}



%%%%%%%%%%%%%%%%%%%%%%

\begin{frame}[allowframebreaks]{Referenser}
  \small
  \printbibliography
\end{frame}

\end{document}

