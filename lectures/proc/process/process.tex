% $Id$
% Author:	Daniel Bosk <daniel.bosk@miun.se>
%\documentclass[handout]{beamer}
\documentclass{beamer}
\usepackage[utf8]{inputenc}
\usepackage[T1]{fontenc}
\usepackage[swedish,english]{babel}
\usepackage{url}
\usepackage{graphicx}
\usepackage{listings}
\usepackage[today,nofancy]{svninfo}
\usepackage[natbib,style=alphabetic,maxbibnames=99,backend=bibtex8]{biblatex}
\addbibresource{literature.bib}
\usepackage[listings]{miunmisc}

\svnInfo $Id$

\mode<presentation>
{
  \usetheme{Frankfurt}
  \setbeamercovered{transparent}
  \usecolortheme{seagull}
}
%\def\newblock{\hskip .11em plus .33em minus .07em}
\setbeamertemplate{footline}{\insertframenumber}

\title{%
  Processes
}
\author{Daniel Bosk\footnote{%
	\tiny
  This work is licensed under the Creative Commons Attribution-ShareAlike 3.0 
  Unported license.
	To view a copy of this license, visit 
	\url{http://creativecommons.org/licenses/by-sa/3.0/}.
}}
\institute[MIUN ICS]{%
  Department of Information and Communication Systems (ICS),\\
  Mid Sweden University, Sundsvall.
}
\date{\svnId}


\AtBeginSection[]{%
  \begin{frame}<beamer>{Overview}
		\tableofcontents[currentsection]
	\end{frame}
}

\begin{document}

\begin{frame}
  \titlepage
\end{frame}

\begin{frame}{Overview}
	\tableofcontents
	% You might wish to add the option [pausesections]
\end{frame}

\begin{frame}{Literature}
  % $Id$
This lecture gives an overview of the technicalities of the file system.
This is covered by Chapter 10 ``File System'' in \cite{Silberschatz2009osc}, or 
Chapter 11 ``File-System Interface'' in 
\cite{Silberschatz2013osc,Silberschatz2013intl}.

\end{frame}


% Since this a solution template for a generic talk, very little can
% be said about how it should be structured. However, the talk length
% of between 15min and 45min and the theme suggest that you stick to
% the following rules:  

% - Exactly two or three sections (other than the summary).
% - At *most* three subsections per section.
% - Talk about 30s to 2min per frame. So there should be between about
%   15 and 30 frames, all told.


\section[Concept]{Process Concept}

\subsection{What is a process?}

\begin{frame}{\insertsubsectionhead}
  \begin{itemize}
    \item Process is the general term for CPU-activity.
    \item The basic executing entity in an operating system (OS).
    \item Process is a program in execution.
    \item Program is not a process:
      \begin{itemize}
        \item Run the same program several times will spawn many unique 
          processes.
      \end{itemize}
  \end{itemize}
\end{frame}

\begin{frame}{\insertsubsectionhead}
  \begin{figure}
    \includegraphics[height=0.7\textheight]{proc.pdf}
    \caption{An schematic of an in-memory process.
      Image: \cite[Fig.\ 3.1, p.\ 102]{Silberschatz2009osc}.}
  \end{figure}
\end{frame}

\subsection{In the OS}

\begin{frame}{\insertsubsectionhead}
  \begin{itemize}
    \item Must share resources fairly among all processes.
    \item OS has to keep track of all processes.
    \item OS uses a structure called Process Control Block (PCB) for this.
  \end{itemize}
\end{frame}

\begin{frame}{\insertsubsectionhead}{PCB}
  \begin{itemize}
    \item Process state
    \item Program counter
    \item CPU-registers
    \item Scheduling information
    \item Memory-information
    \item Accounting information
    \item I/O-information
  \end{itemize}
\end{frame}

\begin{frame}{\insertsubsectionhead}
  \begin{description}
    \item[State] What state the process is currently in:
      \begin{itemize}
        \item New,
        \item Running,
        \item Waiting,
        \item Ready, and
        \item Terminated.
      \end{itemize}
  \end{description}
\end{frame}

\begin{frame}{\insertsubsectionhead}
  \begin{figure}
    \includegraphics[width=\textwidth]{states.pdf}
    \caption{State-transition diagram.
      Image: \cite[p.\ 103]{Silberschatz2009osc}.}
  \end{figure}
\end{frame}

\begin{frame}{\insertsubsectionhead}
  \begin{figure}
    \includegraphics[height=0.7\textheight]{pcb.pdf}
    \caption{The PCB.
      Image: \cite{Silberschatz2009osc}.}
  \end{figure}
\end{frame}

\begin{frame}{\insertsubsectionhead}
  \begin{description}
    \item[Program counter] Where in the program code the process is currently 
      executing.

    \item[CPU-registers] A copy of the values in all CPU-registers, as the 
      CPU-registers are used by the currently executing process.

    \item[Scheduling info] Priorities, scheduling queues.

    \item[Memory-info] Base and limit registers, page tables.

    \item[Accounting] How much CPU-time spent, how much I/O-time spent, time 
      left of time-slice, process ID.

    \item[I/O-info] Allocated I/O-devices, open files.
  \end{description}
\end{frame}

\section[Scheduling]{Process Scheduling}

\subsection{What is process scheduling?}

\begin{frame}{\insertsubsectionhead}
  \begin{itemize}
    \item In a multiprogramming system, the main idea is to use the resources 
      fully -- never idly waiting around.

    \item When one process is waiting for e.g.\ I/O, let another process 
      execute on the CPU.

    \item How to manage this?
      By using a scheduler.
  \end{itemize}
\end{frame}

\begin{frame}{\insertsubsectionhead}
  \begin{figure}
    \includegraphics[height=0.7\textheight]{switch.pdf}
    \caption{Context switch from one process to another.
      Image: \cite{Silberschatz2009osc}.}
  \end{figure}
\end{frame}

\subsection{Schedulers}

\begin{frame}{\insertsubsectionhead}
  \begin{itemize}
    \item There are three types of schedulers.
      \begin{itemize}
        \item Long-term scheduler,
        \item Mid-term scheduler, and
        \item Short-term scheduler.
      \end{itemize}

    \item These are used by different systems, and not necessarily in this 
      structure and by these names.

    \item E.g.\ PC:s usually only have a short-term scheduler.
  \end{itemize}
\end{frame}

\begin{frame}{\insertsubsectionhead}
  \begin{description}
    \item[Long-term scheduler] Comes from purely multiprogramming systems.
      Jobs are submitted to the system, they are stored on disk.
      Eventually the long-term scheduler selects a job from disk and loads it 
      into memory.

    \item[Short-term scheduler] (Also CPU-scheduler) This scheduler chooses 
      among the processes in memory which is to be allowed to execute on the 
      CPU.
      If one process blocks for I/O, then the short-term scheduler quickly 
      chooses another process and allocates the CPU to that new one.

    \item[Medium-term scheduler] This scheduler is used for swapping out 
      processes to disk, to free some more memory for the currently executing 
      processes.
  \end{description}
\end{frame}

\begin{frame}{\insertsubsectionhead}
  \begin{figure}
    \includegraphics[height=0.7\textheight]{queues.pdf}
    \caption{Different queues with some PCBs in them.
      Image: \cite{Silberschatz2009osc}.}
  \end{figure}
\end{frame}

\begin{frame}{\insertsubsectionhead}
  \begin{figure}
    \includegraphics[height=0.7\textheight]{ready.pdf}
    \caption{An illustration of a process's execution.
      Image: \cite{Silberschatz2009osc}.}
  \end{figure}
\end{frame}

\section[Operations]{Operations on Processes}

\subsection{How does processes come into being?}

\begin{frame}{\insertsubsectionhead}
  \begin{itemize}
    \item Upon boot, the kernel is loaded.
    \item The kernel in turn starts the first process: init, with PID 0.
    \item A process can spawn other processes, i.e.\ children, and thus being 
      their parent.
    \item This can be done with the fork(2) system call.
  \end{itemize}
\end{frame}

\begin{frame}{\insertsubsectionhead}
  \begin{figure}
    \includegraphics[width=\textwidth]{proctree.pdf}
    \caption{An example of a process tree from a Solaris system.
      Image: \cite{Silberschatz2009osc}.}
  \end{figure}
\end{frame}

\begin{frame}{\insertsubsectionhead}
  \begin{figure}
    \includegraphics[width=\textwidth]{fork.pdf}
    \caption{An illustration of the UNIX fork(2) system call.
      Image: \cite{Silberschatz2009osc}.}
  \end{figure}
\end{frame}

\begin{frame}{\insertsubsectionhead}
  \begin{itemize}
    \item The parent then waits for its children to finish, using the wait(2) 
      system call.

    \item An alternative is to kill the children, using the kill(2) system 
      call.

    \item If the parent exits, using the exit(2) system call, then the children 
      are assigned a new parent, namely the special init process.
  \end{itemize}
\end{frame}

\begin{frame}{\insertsubsectionhead}
  \begin{itemize}
    \item We also have the exec(2) system call, which replaces the calling 
      process with another process.
  \end{itemize}
\end{frame}

\section[IPC]{Interprocess Communication (IPC)}

\subsection{Independent vs cooperating}

\begin{frame}{\insertsubsectionhead}
  \begin{itemize}
    \item A process can be independent if it works on its own.
    \item If it doesn't work on its own, it's cooperating, i.e.\ it can affect 
      other processes behaviour.
  \end{itemize}
\end{frame}

\begin{frame}{\insertsubsectionhead}{Reasons for cooperating}
  \begin{description}
    \item[Information sharing] Concurrent access to the same file by several 
      users.

    \item[Computation speed-up] Break a given task into subtasks and let 
      different processes execute the different subtasks, the collect the 
      results.

    \item[Modularity] Replacing one process if we want to do one part 
      differently.
  \end{description}
\end{frame}

\subsection{Different types of IPC}

\begin{frame}{\insertsubsectionhead}
  \begin{figure}
    \includegraphics[height=0.7\textheight]{ipc.pdf}
    \caption{Two categories of IPC: (a) message-passing, (b) shared memory.
      Image: \cite{Silberschatz2009osc}.}
  \end{figure}
\end{frame}

\begin{frame}{\insertsubsectionhead}
  \begin{itemize}
    \item Several ways to implement message-passing.
    \item A few examples are:
      \begin{itemize}
        \item Pipes,
        \item Local procedure calls (LPC),
        \item Sockets,
        \item RPC (uses sockets),
      \end{itemize}
  \end{itemize}
\end{frame}

\begin{frame}{\insertsubsectionhead}
  \begin{figure}
    \includegraphics[height=0.7\textheight]{rpc.pdf}
    \caption{An example of an RPC.
      Image: \cite{Silberschatz2009osc}.}
  \end{figure}
\end{frame}


\section[Test C src]{Testing the C examples in the book}

\subsection{Compiling}

\begin{frame}{\insertsubsectionhead}
  \begin{itemize}
    \item Download source from website, or just write it in any text editor 
      (same as for Python).

    \item Go to the directory where the source file is stored.

    \item Compile the source code into executable code.

    \item Run!
  \end{itemize}
\end{frame}

\begin{frame}[fragile]{\insertsubsectionhead}
  \begin{terminal}
$ ${EDITOR} example.c
$ cc -o example example.c
[some output]
$ ./example
  \end{terminal}
\end{frame}


%%%%%%%%%%%%%%%%%%%%%%

\begin{frame}[allowframebreaks]{Referenser}
  \small
  \printbibliography
\end{frame}

\end{document}

