% $Id$
% Author:	Daniel Bosk <daniel.bosk@miun.se>
%\documentclass[handout]{beamer}
\documentclass{beamer}
\usepackage[utf8]{inputenc}
\usepackage[T1]{fontenc}
\usepackage[swedish,english]{babel}
\usepackage{url}
\usepackage{varioref,prettyref}
\usepackage{graphicx}
\usepackage{booktabs}
\usepackage{listings}
\usepackage[today,nofancy]{svninfo}
\usepackage[natbib,style=alphabetic,maxbibnames=99,backend=bibtex8]{biblatex}
\addbibresource{literature.bib}
\usepackage[listings,varioref,prettyref]{miunmisc}

\svnInfo $Id$

\mode<presentation>
{
  \usetheme{Frankfurt}
  \setbeamercovered{transparent}
  \usecolortheme{seagull}
}
%\def\newblock{\hskip .11em plus .33em minus .07em}
\setbeamertemplate{footline}{\insertframenumber}

\title{%
  Synchronisation
}
\author{Daniel Bosk\footnote{%
	\tiny
  This work is licensed under the Creative Commons Attribution-ShareAlike 3.0 
  Unported license.
	To view a copy of this license, visit 
	\url{http://creativecommons.org/licenses/by-sa/3.0/}.
}}
\institute[MIUN ICS]{%
  Department of Information and Communication Systems (ICS),\\
  Mid Sweden University, Sundsvall.
}
\date{\svnId}


\AtBeginSection[]{%
  \begin{frame}<beamer>{Overview}
		\tableofcontents[currentsection]
	\end{frame}
}

\begin{document}

\begin{frame}
  \titlepage
\end{frame}

\begin{frame}{Overview}
	\tableofcontents
	% You might wish to add the option [pausesections]
\end{frame}

\begin{frame}{Literature}
  % $Id$
This lecture gives an overview of the technicalities of the file system.
This is covered by Chapter 10 ``File System'' in \cite{Silberschatz2009osc}, or 
Chapter 11 ``File-System Interface'' in 
\cite{Silberschatz2013osc,Silberschatz2013intl}.

\end{frame}


% Since this a solution template for a generic talk, very little can
% be said about how it should be structured. However, the talk length
% of between 15min and 45min and the theme suggest that you stick to
% the following rules:  

% - Exactly two or three sections (other than the summary).
% - At *most* three subsections per section.
% - Talk about 30s to 2min per frame. So there should be between about
%   15 and 30 frames, all told.

\section{Synchronisation}

\subsection{Where does the problem arise?}

\begin{frame}[fragile]{\insertsubsectionhead}
  \begin{source}[language=Python]
i = 0
while (True):
  i = i + 1
  \end{source}
\end{frame}

\begin{frame}{\insertsubsectionhead}
  \begin{itemize}
    \item When we have a preemptive scheduler or multiprocessor system, 
      multiple processes can try to use the same resource at the same time.

    \item When the outcome of the execution depends on the order of execution 
      of the processes we say we have a race condition.

  \end{itemize}
\end{frame}

\subsection{The Critical-Section Problem}

\begin{frame}{\insertsubsectionhead}
  \begin{itemize}
    \item We can sort out the parts of the code of a program which are 
      suceptible to race conditions, these parts are called the critical 
      sections.

    \item The code just before the critical section is called the entry 
      section.

    \item The code in the end of the critical section is the exit section.

    \item The code after the critical section is the remainder section.

    \item The Critical-Section Problem is about the design of a protocol for 
      several processes to use for cooperation around their critical sections.

  \end{itemize}
\end{frame}

\begin{frame}{\insertsubsectionhead}
  To solve the Critcal-Section Problem an algorithm (protocol) must fulfull the 
  following requirements:
  \begin{enumerate}
    \item Mutual exclusion.
      If a process \(P_i\) is executing in its critical section, no other 
      process may do so.
    \item Progress.
      If no process is executing in its critical section and some processes 
      wish to enter their critical sections, then only those processes 
      \emph{not} in their remainder sections can participate in the decision.
    \item Bounded waiting.
      There exits a bound for how long a process which has requested to enter 
      its critical section may have to wait before allowed to enter.
  \end{enumerate}
\end{frame}

\subsection{Tools to solve the problem}

\begin{frame}{\insertsubsectionhead}
  \begin{itemize}
    \item Peterson's solution.
    \item Locks.
    \item Semaphores.
  \end{itemize}
\end{frame}

\begin{frame}[fragile]{\insertsubsectionhead}{Peterson's Solution}
  \begin{source}[language=Python]
while (True):
  flag[i] = True
  turn = j
  while ( flag[j] and turn == j ):
    pass

  # critical section

  flag[i] = False

  # remainder section
  \end{source}
\end{frame}

\begin{frame}{\insertsubsectionhead}
  \begin{itemize}
    \item On modern computer architectures we need special hardware 
      instructions to help us, e.g.\ TestAndSet or AtomicSwap.
  \end{itemize}
\end{frame}

\begin{frame}[fragile]{\insertsubsectionhead}{TestAndSet}
  \begin{source}[language=Python]
# lock = [False]
while ( True ):
  while ( TestAndSet( lock ) ):
    pass

  # critical section

  lock = False

  # remainder section
  \end{source}
\end{frame}

\begin{frame}[fragile]{\insertsubsectionhead}{AtomicSwap}
  \begin{source}[language=Python]
# lock = [False]
while ( True ):
  key = [True]
  while ( key[0] ):
    AtomicSwap( lock, key ) )

  # critical section

  lock = False

  # remainder section
  \end{source}
\end{frame}

\begin{frame}[fragile,allowframebreaks]{\insertsubsectionhead}{Bounded waiting}
  \begin{source}[language=Python]
# lock = [False]
while ( True ):
  waiting[i] = True
  key = True
  while ( waiting[i] and key[0] ):
    key = TestAndSet( lock ) )

  waiting[i] = False

  # critical section

  j = ( i + 1 ) \% n
  while ( ( j != i ) and not waiting[j] ):
    j = ( j + 1 ) \% n

  if ( j == i ):
    lock = False
  else:
    waiting[j] = False

  # remainder section
  \end{source}
\end{frame}

\begin{frame}[fragile]{\insertsubsectionhead}{Semaphores}
  \begin{itemize}
    \item Use two operations \code{wait()} and \code{signal()}.

    \item A binary valued semaphore is called a mutex lock, since it provides 
      mutual exclusion.

    \item We have also counting semaphores.

  \end{itemize}
\end{frame}

\begin{frame}[fragile]{\insertsubsectionhead}{Semaphores}
  \begin{source}[language=Python]
def wait( S ):
  while ( S <= 0 ):
    pass
  S -= 1
  \end{source}
  \begin{source}[language=Python]
def signal( S ):
  S += 1
  \end{source}
\end{frame}

\begin{frame}[fragile]{\insertsubsectionhead}{Semaphores}
  \begin{source}[language=Python]
while ( True ):
  wait( mutex )

  # critical section

  signal( mutex )

  # remainder section
  \end{source}
\end{frame}



\section{Deadlocks}

\subsection{The requirements}

\begin{frame}{\insertsubsectionhead}
  For a deadlock to occur, the following requirements must be fulfilled:
  \begin{enumerate}
    \item Mutual exclusion.
    \item Hold and wait.
    \item No preemption.
    \item Circular wait.
  \end{enumerate}
  The converse, to prevent deadlocking we must guarantee at least one of the 
  above requirements is not fulfilled at any time.
\end{frame}

\begin{frame}{\insertsubsectionhead}
  \begin{figure}
    \includegraphics[height=0.7\textheight]{deadtruck.pdf}
    \caption{A deadlock.
      Image: \cite{Silberschatz2013osc}.}
  \end{figure}
\end{frame}

\begin{frame}{\insertsubsectionhead}
  \begin{figure}
    \includegraphics[height=0.7\textheight]{deadlock.pdf}
    \caption{Another deadlock.
      Image: \cite{Silberschatz2013osc}.}
  \end{figure}
\end{frame}

\begin{frame}{\insertsubsectionhead}
  \begin{figure}
    \includegraphics[height=0.7\textheight]{nodeadlock.pdf}
    \caption{A similar situation which is not a deadlock.
      Image: \cite{Silberschatz2013osc}.}
  \end{figure}
\end{frame}

\subsection{Dining Philosophers Problem}

\begin{frame}{\insertsubsectionhead}
  \begin{figure}
    \includegraphics[height=0.7\textheight]{dining.pdf}
    \caption{The setup of the Dining Philosophers Problem.
      Image: \cite{Silberschatz2013osc}.}
  \end{figure}
\end{frame}



%%%%%%%%%%%%%%%%%%%%%%

\begin{frame}[allowframebreaks]{Referenser}
  \small
  \printbibliography
\end{frame}

\end{document}

